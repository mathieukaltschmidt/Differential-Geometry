\documentclass{standalone}
\usepackage{tikz} 
\usetikzlibrary{shapes}
\usepackage{pgfplots}
\usepgfplotslibrary{fillbetween}
\usepackage[active,tightpage]{preview}  %generates a tightly fitting border around the work
\PreviewEnvironment{tikzpicture}
\setlength\PreviewBorder{2mm}
\usepackage{xcolor}


\begin{document}
		
\begin{tikzpicture}[scale=5]
	
%erstes Bild	
	
	%Füllen der Grundfläche
	\path[name path=Grundfläche,fill,cyan!15] (0,0,0) -- (0,0,-1) -- (1.2,0,-1) -- (1.2,0,0) -- (0,0,0);
	
	%Rand
	\draw[thin] (0,0,0) -- (1.2,0,0);
	\draw[thin] (0,0,0) -- (0,0,-1);
	\draw[thin] (0,0,-1) -- (1.2,0,-1);
	\draw[thin] (1.2,0,-1) -- (1.2,0,0);
	
	%Punkte auf Karte
	\filldraw (0.85,0.2)circle (0.2pt);
	\node at (0.85,0.12,0) {$p$};
	\node at (0.55,0.2) {$U$};	
	\node at (1.15,0.2) {$\tilde{U}$};
	\node at (0.85,0.45) {Mannigfaltigkeit $\mathcal{M}$};
	
	\begin{scope} [fill opacity = .4]
    \draw[draw = black] (0.8,0.2) circle (0.159);
    \draw[draw = black] (0.9,0.2) circle (0.159);
    \end{scope}

		
	\end{tikzpicture}
\end{document}