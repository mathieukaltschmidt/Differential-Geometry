\documentclass{standalone}
\usepackage{tikz} 
\usetikzlibrary{shapes,patterns}
\usepackage{pgfplots}
\usepgfplotslibrary{fillbetween}
\usepackage[active,tightpage]{preview}  %generates a tightly fitting border around the work
\PreviewEnvironment{tikzpicture}
\setlength\PreviewBorder{2mm}
\usepackage{xcolor}
\definecolor{myred}{RGB}{196,19,47} 
\definecolor{myblue}{RGB}{0,139,139}


\begin{document}
		
\begin{tikzpicture}[scale=5]
	
%erstes Bild	
	
	
	%Rand
	\draw[thin] (0,0,0) -- (1.2,0,0);
	\draw[thin] (0,0,0) -- (0,0,-1);
	\draw[thin] (0,0,-1) -- (1.2,0,-1);
	\draw[thin] (1.2,0,-1) -- (1.2,0,0);
	
% Sich überlappende Mengen
    \node (p1) at (0.6, 0.07) {};
    \node (p2) at (0.9, 0.17) {};
    \node (p3) at (0.5, 0.27) {};
   
    \node (q1) at (0.7,0.1) {};
    \node (q2) at (0.7,0.25) {};
    \node (q3) at (1.1, 0.25) {};
    \node (q4) at (1.0, 0.1) {};
    \path[pattern=north west lines, pattern color=myblue!50]plot [smooth cycle,tension=1] coordinates {(q1) (q2) (q3)(q4)};
    \path[pattern=north east lines, pattern color=myred!50]plot [smooth cycle,tension=1] coordinates {(p1) (p2) (p3)}; 
       \path[draw, black]plot [smooth cycle,tension=1] coordinates {(q1) (q2) (q3)(q4)};	
    \path[draw, black]plot [smooth cycle,tension=1] coordinates {(p1) (p2) (p3)}; 		
    
 %Punkte auf Karte
    \filldraw (0.75,0.18)circle (0.2pt);
    \node at (0.35,0.1) {$p$};
    \draw[thin, ->] (0.39,0.1) parabola (0.72,0.17);
    \node at (0.42,0.3) {$U$};	
    \node at (1.2,0.3) {$\tilde{U}$};
    \node at (0.75,0.45) {$\mathcal{M}$};
	\end{tikzpicture}
\end{document}