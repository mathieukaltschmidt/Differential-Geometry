\documentclass{standalone}
\usepackage{tikz} 
\usetikzlibrary{shapes,patterns}

\definecolor{myred}{RGB}{196,19,47} 
\definecolor{myblue}{RGB}{0,139,139}

\usepackage{pgfplots}
\usepgfplotslibrary{fillbetween}
\usepackage[active,tightpage]{preview}  %generates a tightly fitting border around the work
\PreviewEnvironment{tikzpicture}
\setlength\PreviewBorder{2mm}
\usepackage{xcolor}

\begin{document}
  \begin{tikzpicture}
  
  	% Mannigfaltigkeit
	%Füllen der Grundfläche
	%\path[name path=Grundfläche,fill,myblue!30] (0,0,0) -- (0,0,-1) -- (1.2,0,-1) -- (1.2,0,0) -- (0,0,0);
	%Rand
	\draw[thin] (-1,0) -- (4.5,0);
	\draw[thin] (4.5,0) -- (5.5,2);
	\draw[thin] (5.5,2) -- (0,2);
	\draw[thin] (0,2) -- (-1,0);  	
  	% Mannigfaltigkeit als 'blob'
	%    \node (p1) at (-0.4, 0) {};
	%    \node (p2) at (0.5, 2.4) {};	
	%    \node (p3) at (4.5, 2) {};
	%    \node (p4) at (4, -0.2) {};
	%    \path[draw]plot [smooth cycle,tension=1] coordinates {(p1) (p2) (p3) (p4)};
	
	%Kurve
	\draw[-] (1,1.3) [out=280, in=100] to  (3.9,0.9);
	\filldraw (1,1.3)circle (1pt);
	\filldraw (3.9,0.9)circle (1pt);

	% Ebene 1
	\draw[thin] (0,3.08) -- (0.5,3.28);
	\draw[thin] (0.5,3.28) -- (0.5,4.68);
	\draw[thin] (0.5,4.68) -- (0,4.48);
	\draw[thin] (0,4.48) -- (0,3.08);
	\filldraw (0.25,3.85) circle (1pt);
	% Ebene 2
	\draw[thin] (1,3.31) -- (1.5,3.51);
	\draw[thin] (1.5,3.51) -- (1.5,4.91);
	\draw[thin] (1.5,4.91) -- (1,4.71);
	\draw[thin] (1,4.71) -- (1,3.31);
	\filldraw (1.25,4.08) circle (1pt);
	% Ebene 3
	\draw[thin] (2,3.70) -- (2.5,3.90);
	\draw[thin] (2.5,3.90) -- (2.5,5.30);
	\draw[thin] (2.5,5.30) -- (2,5.10);
	\draw[thin] (2,5.10) -- (2,3.70);
	\filldraw (2.25,4.47) circle (1pt);
	% Ebene 4
	\draw[thin] (3,4.09) -- (3.5,4.29);
	\draw[thin] (3.5,4.29) -- (3.5,5.69);
	\draw[thin] (3.5,5.69) -- (3,5.49);
	\draw[thin] (3,5.49) -- (3,4.09);
	\filldraw (3.25,4.86) circle (1pt);
	% Ebene 5
	\draw[thin] (4,4.37) -- (4.5,4.57);
	\draw[thin] (4.5,4.57) -- (4.5,5.97);
	\draw[thin] (4.5,5.97) -- (4,5.77);
	\draw[thin] (4,5.77) -- (4,4.37);
	\filldraw (4.25,5.14) circle (1pt);
	
	% Kurven auf E_p
	\draw[thick, dotted, -] (-0.6,4.5) [out=280, in=100] to  (5.3,4.5);
	\filldraw  (-0.6,4.5) circle (1pt);
	\filldraw (5.3,4.5) circle (1pt);
	
	% Vektoren	
	\draw[thick,color=myred,->] (0.25,3.85) -- (0.3,4.5);
	\draw[thick,color=myred,->] (1.25,4.08) -- (1.4,4.6);
	\draw[thick,color=myred,->] (2.25,4.47)  -- (2.4,4.05);
	\draw[thick,color=myred,->] (3.25,4.86) -- (3.1,4.35);
	\draw[thick,color=myred,->] (4.25,5.14) -- (4.15,5.64);
	
	% text	
	\node at (0,0.5) {$\mathcal{M}$};
	% arrow
	\node at (1.5,5.9) {$E_p$};
	\draw[<-] (0.3,4.7) [out=100, in=-90] to  (1.5,5.7);

  \end{tikzpicture}
\end{document}


	
