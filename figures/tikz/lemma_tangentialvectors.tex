\documentclass[border=5mm]{standalone}
\usepackage{tikz}
\usetikzlibrary{shapes}
\usepackage{pgfplots}
\usepgfplotslibrary{fillbetween}
\usepackage[active,tightpage]{preview} 
\usepackage{amssymb}
\usetikzlibrary{positioning,arrows}
\usepackage[active,tightpage]{preview}  %generates a tightly fitting border around the work
\PreviewEnvironment{tikzpicture}
\setlength\PreviewBorder{2mm}
\usepackage{xcolor}

\tikzset{
  state/.style={circle,draw,minimum size=6ex},
  arrow/.style={-latex, shorten >=1ex, shorten <=1ex}}
  
\begin{document}
		
\begin{tikzpicture}[scale=8]
\tikzstyle{every node}=[font=\LARGE]	
% Fläche (Rechteck)
	\path[name path=Grundfläche,fill,cyan!15] (0,0,0) -- (0,0,-1) -- (1.2,0,-1) -- (1.2,0,0) -- (0,0,0);
	\draw[] (0,0,0) -- (1.2,0,0);
	\draw[] (0,0,0) -- (0,0,-1);
	\draw[] (0,0,-1) -- (1.2,0,-1);
	\draw[] (1.2,0,-1) -- (1.2,0,0);
	
%Menge 1 in Rechteck
    \node (p1) at (0.4, 0.1) {};
    \node (p2) at (0.9, 0.3) {};
    \node (p3) at (1.0, 0.1) {};
    \path[draw, black]plot [smooth cycle,tension=1] coordinates {(p1) (p2) (p3)};
    
%Menge 2 in Rechteck
    \node (p4) at (0.6, 0.1) {};
    \node (p5) at (0.7, 0.25) {};
    \node (p6) at (0.9, 0.1) {};
    \path[draw,black]plot [smooth cycle,tension=1] coordinates {(p4) (p5) (p6)};
    
% Menge Rechts
    \node (q1) at (2.3, 0) {};
    \node (q2) at (2.3, 0.4) {};
    \node (q3) at (2.9, 0.2) {};
    \path[fill,cyan!15]plot [smooth cycle,tension=1] coordinates {(q1) (q2) (q3)};
    \path[draw, black]plot [smooth cycle,tension=1] coordinates {(q1) (q2) (q3)};
    
% Menge Rechts innen
    \node (q1) at (2.4, 0.1) {};
    \node (q2) at (2.4, 0.3) {};
    \node (q3) at (2.8, 0.2) {};
    \path[draw, black]plot [smooth cycle,tension=1] coordinates {(q1) (q2) (q3)};

% Beschriftung
	\filldraw (0.7,0.15)   circle (0.2pt);
	\node at (0.74,0.14)   {$p$};
	\node at (0.7,0.3)   {$U$};
	\node at (0.95,0.2)   {$U'$};		
	\node at (0.8,0.45)   {$\mathcal{M}$};
	\node at (2.5,0.45)   {$V$};
	\node at (2.7,0.3)   {$V'$};		
	\node at (1.15,-0.4)   {$\mathbb{R}$};	
	\filldraw (2.5,0.2)   circle (0.2pt);
	\node at (2.58,0.2)   {$x(p)$};	
	
%Beschriftung der Abbildungen
\node at (1.5,0.5)   {$x$};
\node at (0.75,-0.2)   {$f$};	
\node at (1.9,-0.2)   {$\psi = f \circ x^{-1}$};	
% Pfeil für Abbildungen	
	\draw [arrow, bend angle=30, bend left]  (0.7,0.15)  to (2.3,0.3) ;
	\draw [arrow, bend angle=30, bend right]  (0.7,0.15)  to (1.1,-0.4) ;
	\draw [arrow, bend angle=30, bend left]  (2.4,0.1)  to (1.2,-0.4) ;
	%\node at (5,3.4)   {$x_\alpha$};
	%\draw [arrow, bend angle=45, bend right]  (0.5,0)  to (7.5,-5) ;
	%\node at (5,-4.2)   {$x_\beta$};	
	
	
	
	
	\end{tikzpicture}	
\end{document}