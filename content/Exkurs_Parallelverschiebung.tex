\subsection{Parallelverschiebung}
Auch für das in der Vorlesung eingeführte Konzept der Parallelverschiebung wollen wir nachfolgend ein explizites Beispiel liefern. \\
\phantom{.}\\
Sei $\gamma(t)$ die Kurve, die durch die Gleichungen $x = t, y = t$ und $z = t$  definiert wird. Grundlage der Berechnung ist die zuvor bereits verwendete Metrik
\begin{align*}
g= (1+x^2)\ \dd x\otimes \dd x + \dd y \otimes \dd y + \e^z \ \dd z \otimes \dd z)
\end{align*} 
Wie sieht der Paralleltransport entlang $\gamma$ aus? \\

Der Anfangsvektor sei $\left(v_1(0), v_2(0), v_3(0) \right)^T$. Die Formel für den Paralleltransport lautet: 
\begin{align*}
\frac{\dd v(t)}{\dd t} = \sum_{j, k} \Gamma_{jk}^i \frac{\dd x^j}{\dd t} v^k =0
\end{align*}
Komponentenweises Einsetzen unter Berücksichtigung der im vorherigen Abschnitt berechneten Christoffel-Symbole liefert ein System aus drei Differentialgleichungen erster Ordnung:
\begin{align*}
\begin{cases}
\frac{\dd v_1(t)}{\dd t} + \frac{t}{1+t^2}\underbrace{\frac{\dd x}{\dd t}}_{= 1} v_1(t) = 0 \\
\frac{\dd v_2(t)}{\dd t} = 0 \\
\frac{\dd v_2(t)}{\dd t} + \frac{1}{2} \underbrace{\frac{\dd z}{\dd t}}_{= 1} v_3(t)= 0 
\end{cases}
\end{align*}

Die Lösungen dieser Differentialgleichungen ergeben sich zu:
\begin{align*}
	\begin{cases}
		v_1(t) &= \frac{v_1(0)}{\sqrt{1+t^2}}\\
	v_2(t) &= v_2(0) \\
	 v_3(t)& = v_3(0)\e^{-\frac{t}{2}}
	\end{cases}
\end{align*}
Seien $v(0) = \left(v_1(0), v_2(0), v_3(0) \right)^T$ sowie $w(0) = \left(w_1(0), w_2(0), w_3(0) \right)^T$ gegeben, dann gilt für die Vektorfelder, welche durch den Paralleltransport entlang $\gamma$ entstehen:
\begin{align*}
g(v(t),w(t)) = \underbrace{\sum_{i=1}^{3} v_i(0)w_i(0)}_{\text{const.}}
\end{align*}
Es ergibt sich schlussendlich:
\begin{align*}
x g(v,w) = g(\nabla_xv,w) + g(v,\nabla_xw) = 0.
\end{align*}
