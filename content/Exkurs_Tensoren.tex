\section{Tensoren \& Tensorrechnung}
In diesem Abschnitt seien grundsätzlich alle $V$ endlichdimensionale Vektorräume. \\
\begin{align*}
\mathcal{L}(V_1,\dots, V_k, W) = &\text{ Menge aller multilinearen Abbildungen} \\
&\phantom{ = } V_1 \times V_2 \times \dots \times V_k \longrightarrow W
\end{align*}

\begin{satz}
Für diesen Fall existiert ein Isomorphismus
\begin{align*}
V_1 \otimes \cdots \otimes V_k \cong \mathcal{L}(V_1^*, \dots, V_k^*, \R)
\end{align*}
\end{satz}

\begin{bem}
\begin{align*}
V^{**} &\cong V \\
V_1^* \otimes \cdots \otimes V_k^* &\cong \mathcal{L}(V_1, \dots, V_k, \R)
\end{align*}
\end{bem}

\begin{defs}
Ein kovarianter $k$-Tensor auf $V$ ist ein Element in $\underbrace{V^*\otimes\cdots\otimes V^*}_{\text{k-mal}}$ , das heißt es handelt sich um eine multilineare Abbildung: $ V \times \cdots \times V \longrightarrow \R$
\end{defs}
\textbf{Beispiel:} Die Determinante ist ein kovarianter $k$-Tensor in $\R^k$.
\begin{defs}
Ein kontravarianter $k$-Tensor auf $V$ ist ein Element in $\underbrace{V\otimes\cdots\otimes V}_{\text{k-mal}}$ , das heißt es handelt sich um eine multilineare Abbildung: $ V^*\times \cdots \times V^* \longrightarrow \R$
\end{defs}
Typischerweise treten gemischte Tensoren auf, das heißt Elemente in:
\begin{align*}
\operatorname{T}^{(n, l)}V = \underbrace{V\otimes\cdots\otimes V}_{\text{n-mal}} \ \otimes \  \underbrace{V^*\otimes\cdots\otimes V^*}_{\text{l-mal}}
\end{align*}

\subsubsection*{Tensoren auf Mannigfaltigkeiten}
$\tor^{(r, s)}T\mfk  := \bigsqcup_{p\in\mfk} \tor^{(r, s)}(T_p\mfk)$

\begin{defs}
Ein Tensorfeld vom Typ $(r,s)$ ist ein Schnitt von $\tor^{(r, s)}T\mfk$.
\end{defs}

\textbf{Übung:} Bestimmen sie den Typ der folgenden Tensoren:
\begin{itemize}
\item Vektorfelder
\item $1$-Formen
\item Torsion
\item Krümmungstensor
\end{itemize}