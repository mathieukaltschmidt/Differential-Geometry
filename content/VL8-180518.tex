% !TeX root = ..//diffgeo_main.tex

Als nächstes möchten wir Lemma \ref{lem:lokalisierung1} beweisen.
Allerdings können wir gleich etwas allgemeineres beweisen wodurch Lemma \ref{lem:lokalisierung1} sofort klar ist.

\begin{lem}
\label{lem:tensorielllokalisierung}
Sei $\mathcal{L}: \Gamma (E) \to \Gamma(E')$ eine tensorielle Abbildung. 
Tensoriell bedeuetet hierbei, dass
\begin{align}
\mathcal{L}(\phi s) = \phi \mathcal{L}(s), \quad \forall \phi \in \mathcal{F}(\mfk)
\end{align}
gilt.
Sei weiterhin $p\in \mfk$ und $s, \tilde{s} \in \Gamma(E)$ mit $s(p) = \tilde{s}(p)$, dann gilt
\begin{align}
\mathcal{L}(s)(p) = \mathcal{L}(\tilde{s})(p)
\end{align} 
\end{lem}
Lemma \ref{lem:lokalisierung1} folgt sofort aus diesem Lemma, da die Abbildung
\begin{align}
&\covd_\cdot s : \mathfrak{X}(\mfk) \to \Gamma (E)\\
& x \mapsto \covd_x s
\end{align}
tensoriell ist.

\begin{bew}[Lemma \ref{lem:tensorielllokalisierung}]
Sei $U$ eine Umgebung von $p$ und $\phi = (s_1, \dots, s_k)$ ein lokaler Rahmen auf $U$.
Sei außerdem $\varphi$ eine Bumpfunktion mit $\supp \varphi \subset U$ und $\varphi (p) = 1$.
Wir schreiben 
\begin{align}
s = \sum_{i=1}^k \sigma_i s_i , \quad \tilde{s} = \sum_{i=1}^k \tilde{\sigma}_i s_i
\end{align}
mit $\sigma_i (p) = \tilde{\sigma}_i (p)$.
\begin{align}
\mathcal{L}(s)(p) &= \varphi^2 (p) \mathcal{L}(s)(p)\\
&= \mathcal{L}(\varphi^2 s)(p)\\
&= \sum_{i = 1}^k \mathcal{L}((\varphi \sigma_i)(\varphi s_i)) (p)\\
&= \sum_{i = 1}^k \varphi (p) \sigma_i (p) \mathcal{L}((\varphi s_i)) (p) 
\end{align}
Analog rechnet man mit $\tilde{s}$
\begin{align}
\mathcal{L}(\tilde{s})(p) = \sum_{i = 1}^k \varphi (p) \tilde{\sigma}_i (p) \mathcal{L}((\varphi s_i)) (p) .
\end{align}
Da $\sigma_i = \tilde{\sigma}_i$ und $\varphi(p)=1$ folgt nun die Aussage
\begin{align}
\mathcal{L}(s)(p) = \mathcal{L}(\tilde{s})(p).
\end{align}
\end{bew}

\begin{defs}[Tensorfeld]
Ein Tensorfeld vom Typ $(n, s)$ ist ein glatter Schnitt des Bündels
\begin{align}
T^{n}_{s} (\mfk) = \left( \bigotimes^n_{i=1} T \mfk \right) \otimes \left(\bigotimes^s_{i=1} T^{s*} \mfk\right).
\end{align}
In anderen Worten ist ein Tensorfeld vom Typ $(n,s)$ eine Abbildung
\begin{align}
B : \underbrace{\mathfrak{X}(\mfk) \times \cdots \times \mathfrak{X}(\mfk)}_{s \ \mathrm{mal}} \to \underbrace{\mathfrak{X}(\mfk) \times \cdots \times \mathfrak{X}(\mfk)}_{n \ \mathrm{mal}},
\end{align}
die tensoriell in jedem Argument ist.\\
Lemma \ref{lem:tensorielllokalisierung} sagt uns, dass jede Abbildung $B$ aus einem Vektorfeld kommt.
\end{defs}
An dieser Stelle wollen wir noch einmal kurz einige Fakten über Tensoren sammeln.
Ein Tensor vom Rang $n$ ist:
\begin{align}
t = \sum^n_{i=1} \xi_i \otimes \eta_i
\end{align}
Sei $V$ ein Vektorraum. 
Dann gibt es eine Korrespondenz zwischen
\begin{enumerate}
\item bilineare Abbildung 
\begin{align}
V \times V \to \R
\end{align}
\item Tensoren 
\begin{align}
V^* \otimes V^* = (V \otimes V)^*
\end{align}
\item lineare Abbildungen
\begin{align}
V \otimes V \to \R
\end{align}
\end{enumerate}
wie folgt:
Seien $\xi, \eta \in V^*$ mit $\xi, \eta: V \to \R$.
Dann gibt es die folgende bilineare Abbildung 
\begin{align}
(\xi \otimes \eta)(v, w) = \xi (v) \eta (w).
\end{align}
Allgemeiner hat das Tensorpodukt die folgende Gestalt:
\begin{align}
\left( \bigotimes^n V \right) \otimes \left( \bigotimes^s V^* \right)
\end{align}


Wir kehren nun wieder zu den Tensorfeldern zurück.

\begin{kor}
Sei $B$ ein Tensorfeld vom Typ $(n, s)$, dann induziert $B$ für alle $p$ eine $s$-lineare Abbildung:
\begin{align}
&B_p: T_p \mfk^s \to T_p \mfk^n\\
&(v_1, \dots, \v_s) \mapsto B_p (v_1, \dots, v_s)
\end{align}
\end{kor}
Wir wollen nun wieder zu Zusammenhängen zurückkehren.
\begin{bsp}[Kanonischer Zusammenhang]
Wir wählen die Koordinaten wie folgt:
\begin{align}
s(p) = (p, \sigma(p)), \quad s \in \Gamma(\mfk \times \R^k),
\end{align}
wobei $\sigma = (\sigma_1, \dots, \sigma_k)$ wobei $\sigma_i \in \mathcal{F}(\mfk)$.
Dann ist der kanonische Zusammenhang wie folgt gegeben:
\begin{align}
\covd_x s = (x(\sigma_1), \dots, x(\sigma_k))
\end{align}
\end{bsp}

Sei $\omega$ eine $1$-Form auf $\mfk$ mit Werten in Matrizen $\mathrm{Mat}_{k \times k}(\R)$.
Das bedeutet, dass
\begin{align}
&\omega \in \Gamma(\mathrm{Hom}(T\mfk, \mathrm{Mat}_{k \times k}(\R))) \\
&\omega_p: T\mfk \to \mathrm{Mat}_{k \times k}(\R) \\
&\omega = (\omega{ij})^k_{i,j=1}
\end{align}
$\omega_{ij}$ ist eine $1$-Form auf $\mfk$
\begin{align}
w^{ij}_p :T_p \mfk \to \R.
\end{align}

\begin{defs}
Mit einer $1$-Form kann folgender Zusammenhang definiert werden:
\begin{align}
(\covd^\omega_x s)(p) = (p, x_p(\sigma) + \omega_p(x_p) \sigma(p))
\end{align}
\end{defs}

\begin{satz}
\label{satz:zusammenhangausform}
Sei $(E, \pi, \mfk)$ ein Vektorbündel und $\covd$ ein Zusammenhang auf $E$.
Weiterhin sei $\omega$ eine $1$-Form mit Werten in $\mathrm{Hom}(E, E)$.
Dann ist 
\begin{align}
D^\omega_x s = \covd_x s + \omega(x) s,
\end{align}
ein Zusammenhang auf $E$
\end{satz}
Umgehrt gilt ebenso der folgende Satz:
\begin{satz}
Seien $\covd$ und $\covd'$ zwei Zusammenhänge auf $E$.
Dann definiert 
\begin{align}
\omega(x) s = \covd'_x s - \covd_x s
\end{align}
eine $1$-Form mit Werten in $\mathrm{Hom}(E, E)$.
\end{satz}
\begin{bsp}
Sei $E = \mfk \times \R^k$. 
\begin{align}
\covd^\omega_x s = (p, x_(\sigma) + \omega_p (x_p)\sigma_p)
\end{align}
Aus dem Satz von eben folgt, dass $\covd^\omega_x s$ alle Zusammenhänge auf $E$ sind.
Aus der Lokalisierung folgt
\begin{align}
\phi : U \times \R^k \to E\big\vert_U.
\end{align}
\end{bsp}
Wir stellen uns die Frage wie wir Zusammenhänge mithilfe einer $1$-Form finden können.
Dies ist mithilfe von Zusammenhangsformen bezüglich eines lokalen Rahmens möglich.
Sei $\phi = (s_1, \dots, s_k)$ ein lokaler Rahmen von $E$ über $U$ und sei $x \in \mathfrak{X}(\mfk)$.
Zudem sind $\covd_x s_1, \dots, \covd_x s_k \in \Gamma(E)$.
Diese lassen sich wie folgt darstellen:
\begin{align}
\covd_x s_i = \sum_{i=1}^k \omega_{ij}(x) s_j
\end{align}
Wobei die $1$-Form $\omega_{ij}(x): U \to \R$ glatt und tensoriell in $x$ ist.
Es ist $s = \sum_i \sigma_i s_i$. 
Damit erhält man:
\begin{align}
\covd_x s &= \sum (x(\sigma_i)s_i + \sigma_i + \sigma_i \covd_x s_i)\\
&= \sum_j x(\sigma_j)s_j + \sum_{j=1}^k \sum_{i=1}^k \sigma_i \omega_{ij} s_j\\
&= \sum_j \left[ x(\sigma_j) + \sum:i \sigma_i \omega_{ij}(x) \right] s_j
\end{align}


