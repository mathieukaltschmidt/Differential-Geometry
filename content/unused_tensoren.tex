An dieser Stelle wollen wir noch einmal kurz einige Fakten über Tensoren sammeln.
Ein Tensor vom Rang $n$ ist:
\begin{align}
t = \sum^n_{i=1} \xi_i \otimes \eta_i
\end{align}
Sei $V$ ein Vektorraum. 
Dann gibt es eine Korrespondenz zwischen
\begin{enumerate}
\item bilineare Abbildung 
\begin{align}
V \times V \to \R
\end{align}
\item Tensoren 
\begin{align}
V^* \otimes V^* = (V \otimes V)^*
\end{align}
\item lineare Abbildungen
\begin{align}
V \otimes V \to \R
\end{align}
\end{enumerate}
wie folgt:
Seien $\xi, \eta \in V^*$ mit $\xi, \eta: V \to \R$.
Dann gibt es die folgende bilineare Abbildung 
\begin{align}
(\xi \otimes \eta)(v, w) = \xi (v) \eta (w).
\end{align}
Allgemeiner hat das Tensorpodukt die folgende Gestalt:
\begin{align}
\left( \bigotimes^n V \right) \otimes \left( \bigotimes^s V^* \right)
\end{align}

