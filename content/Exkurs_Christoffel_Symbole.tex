
In diesem Abschnitt wollen wir anhand zweier Beispiele explizite Rechnungen in Koordinaten ausführen. \\
Wir beginnen zunächst mit der Euklidischen Metrik auf $\R^2$:
\begin{align}
g=\dd x^2 + \dd y^2
\end{align}
Wir wollen eine Koordinatentransformation in ebene Polarkoordinaten durchführen:
\begin{align*}
x &\longrightarrow r\cdot \cos\varphi \\
y &\longrightarrow r\cdot \sin\varphi
\end{align*}
Wir erhalten demnach mit Hilfe der Transformationsregel:
\begin{align*}
\dd x &= \frac{\partial x}{\partial r} \dd r + \frac{\partial x}{\partial \varphi} \dd \varphi \\
&= \cos\varphi\dd r - r\sin\varphi\dd\varphi \\
\phantom{.}\\
\dd y &= \sin\varphi\dd r + r\cos\varphi\dd\varphi
\end{align*}
Damit folgt für unsere neue Metrik $g_{\tiny\text{Polar}}$ die Darstellung:
\begin{align*}
g_{\text{Polar}} &= \dd x \otimes \dd x + \dd y \otimes \dd y \\
\phantom{.}\\
&= (\cos\varphi\dd r - r\sin\varphi\dd\varphi) \otimes (\cos\varphi\dd r - r\sin\varphi\dd\varphi) \\ &+ (\sin\varphi\dd r + r\cos\varphi\dd\varphi) \otimes (\sin\varphi\dd r + r\cos\varphi\dd\varphi) \\
\phantom{.}\\
&= \cos^2\varphi \ \dd r \otimes \dd r - r\sin\varphi\cos\varphi \ \dd r \otimes \dd \varphi \\
&- r\sin\varphi\cos\varphi \ \dd \varphi \otimes \dd r + r^2\sin^2\varphi \ \dd\varphi \otimes \dd\varphi \\
&+ \sin^2\varphi \ \dd r \otimes \dd r + r\sin\varphi\cos\varphi \ \dd r \otimes \dd\varphi \\
&+ r \sin\varphi\cos\varphi \ \dd\varphi \otimes \dd r + r^2 \ \dd\varphi \otimes \dd\varphi \\
\phantom{.}\\
&= \dd r \otimes \dd r + r^2 \ \dd\varphi \otimes \dd\varphi
\end{align*}
Die Darstellung der ersten Fundamentalform ergibt sich demnach zu:
\begin{align}
\left(g_{ij}\right)_{\text{Polar}} = 
\begin{pmatrix}
1 & 0 \\
0 & r^2
\end{pmatrix}
\end{align}
\subsection{Christoffel-Symbole}
Wir wollen nun mit dieser Erkenntnis einige Christoffel-Symbole für verschiedene Metriken explizit ausrechnen. \\
Zunächst einmal betrachten wir wieder die \textbf{euklidische Metrik} auf $\R^2$: 
\begin{align*}
g_{ij} = \delta_{ij} = \operatorname{diag}(1,1)
\end{align*}
In diesem Fall verschwinden alle Christoffel-Symbole, da der $\R^2$, allgemeiner auch der $\R^n$ keine Krümmung besitzt, also \textit{flach} ist. WIr erhalten:
\begin{align*}
\Gamma_{ij}^k = 0 \quad \forall i, j ,k
\end{align*}
Dies bedeutet, dass die kovariante Ableitung gleich der "gewöhnlichen"  Ableitung ist, was wir auch erwarten.\\
Kommen wir nun zu der zuvor hergeleiteten Metrik für die \textbf{ebenen Polarkoordinaten}:
\begin{align*}
g= \dd r \otimes \dd r + r^2 \ \dd\varphi \otimes \dd\varphi
\end{align*}
Wir erhalten \ $\frac{\partial}{\partial\varphi} = -y\frac{\partial}{\partial x} +x \frac{\partial}{\partial y}$ \ und damit schließlich:
\begin{align*}
\nabla_{(\frac{\partial}{\partial\varphi})}\frac{\partial}{\partial\varphi}&= \nabla_{(-y\frac{\partial}{\partial x} +x \frac{\partial}{\partial y})}(-y\frac{\partial}{\partial x} +x \frac{\partial}{\partial y}) \\
&=(-y\frac{\partial}{\partial x} +x \frac{\partial}{\partial y})(-y)\frac{\partial}{\partial x} + (-y\frac{\partial}{\partial x} +x \frac{\partial}{\partial y})(x)\frac{\partial}{\partial y} \\
&= -x\frac{\partial}{\partial x} -y\frac{\partial}{\partial x}\\
&= r\frac{\partial}{\partial r}
\end{align*}

Mit diesem Wissen berechnen wir nun explizit die Christoffel-Symbole:
\begin{align*}
\Gamma_{11}^1 &= \frac{1}{2}(1(0+0-0) + 0 \dots) = 0 \\
\Gamma_{11}^1 &= \Gamma_{11}^2 = \Gamma_{12}^1 = \Gamma_{21}^1 = \Gamma_{22}^2 = 0 \\
\Gamma_{12}^2 &= \Gamma_{21}^2 = \frac{1}{2}(r^{-2}\frac{\partial g_{12}}{\partial\varphi} + \frac{\partial g_{22}}{\partial r} - \frac{\partial g_{12}}{\partial\varphi}) = \frac{1}{r} \\
\Gamma_{22}^1 &= -r
\end{align*}

Mit diesem Ergebnis erhalten wir analog wie zuvor:
\begin{align*}
\nabla_{(\frac{\partial}{\partial\varphi})}\frac{\partial}{\partial\varphi} &= \Gamma_{22}^1\frac{\partial}{\partial r} + \Gamma_{22}^2\frac{\partial}{\partial\varphi}\\
&= -r \frac{\partial}{\partial\varphi}
\end{align*}
Zuletzt wollen wir uns eine etwas kompliziertere Metrik auf dem $\R^3$ anschauen:
\begin{align*}
(\R^3, g= (1+x^2)\ \dd x\otimes \dd x + \dd y \otimes \dd y + \e^z \ \dd z \otimes \dd z)
\end{align*}
Daraus folgt direkt für die erste Fundamentalform und ihr Inverses:
\begin{align*}
g_{ij} &= \operatorname{diag}((1+x^2), 1, e^z) \\
g_{ij}^{-1} &= \operatorname{diag}((1+x^2)^{-1}, 1, e^{-z})
\end{align*}
Wir berechnen damit wieder die Christoffel-Symbole mit der bekannten Formel:
\begin{align*}
\Gamma_{ij}^k= \frac{1}{2}\sum_l g^{kl}(\partial_ig_{jl} + \partial_jg_{il}-\partial_lg_{ij})
\end{align*}
Betrachten wir beispielsweise $\Gamma_{12}^3$ fällt auf, dass:
\begin{align*}
\Gamma_{12}^3 &=  \frac{1}{2}\sum_{l=1}^{3} g^{3l}(\partial_1g_{2l} + \partial_2g_{1l}-\partial_lg_{12}) \\
&=  \frac{1}{2}g^{33}(\partial_1g_{23} + \partial_2g_{13}-\partial_3g_{12}) \\
&= 0 \quad \text{, da $g$ diagonal ist.}
\end{align*}
Man findet:
\begin{align*}
\Gamma_{11}^1 &=  \frac{1}{2} g^{11}\underbrace{(2\cdot\partial_1g_{11} -\partial_1g_{11})}_{ = \  \partial_1g_{11} } \\
&= \frac{1}{2} (1+x^2)^{-1}\cdot 2x \\
&= \frac{x}{1+x^2}
\end{align*}
außerdem berechnet man: \quad $\Gamma_{33}^3 = \frac{1}{2}$. \\
\phantom{.} \\
 Alle anderen Christoffel-Symbole sind null!