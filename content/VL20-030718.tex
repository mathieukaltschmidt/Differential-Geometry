\section{Satz von Bonnet-Myers}

Unser nächstes Ziel wird es sein den Satz von Bonnet-Myers zu beweisen.
Als Vorbereitung hierfür benötigen wr die zweite Variation der Energie.\\
\textbf{Zur Erinnerung}:
Ist $c_s$ eine glatte Variation von $c: [a, b] \to \mfk$ mit Variationsfeld
$\xi$.
dann ist die erste Variations der Energie, wie folgt gegeben:
\begin{align}
    \dv{s} E(c_s) \eval_{s=0} = - \int^b_a \langle \xi , \covd_t \dot{c} \rangle \dd t
    + \langle \xi , \dot{c} \rangle \eval^b_a
\end{align}

\begin{satz}[Zweite Variation der Energie]
Sei $c: [a, b]\to \mfk$ ein Geodätische und $c_s$ eine Variation von $c$,
mit Variationsfeld $\xi$.
Dann gilt:
\begin{align}
    \dv[2]{s} E(c_s) \eval_{y=0} = \int^b_a  \langle \covd_t \pdv{c_s}{s} , 
    \pdv{c_s}{t} \rangle \dd t
\end{align}

\end{satz}

\begin{bew}
Wir wissen bereits:
\begin{align*}
    \dv{s} E(c_s) \eval_{s=0} = \int^b_a \langle \covd_t \pdv{c_s}{s} , 
    \pdv{c_s}{s} \rangle \dd t
\end{align*}
Damit erhält man:
\begin{align*}
\dv[2]{s}E(c_s)\eval_{s=0}&=\int^b_a \langle\covd_s\covd_t\pdv{c_s}{s},\pdv{c_s}{t}\rangle + 
\langle \covd_t \xi , \covd \pdv{c_s}{t} \eval_{s=0} \rangle \dd t \\
&= \int^b_a \underbrace{\langle\covd_t\covd_s\pdv{c_s}{s}\eval_{s=0},\dot{c}\rangle}_{\ast}
+ \int^b_a \langle \curv (\xi , \dot{c})\xi , \dot{c}
+ \int^b_a \langle\covd_t\xi , \covd_t \xi\rangle\dd t 
\end{align*}

Betrachte $\ast$:
\begin{align*}
\ast &= \int^b_a\left( \pdv{t} \langle \covd_s \pdv{c_s}{s}\eval_{s=0}, c\pdv \right
- \underbrace{\langle\covd_s\pdv{c_s}{s}\eval_{s=0} , 
\covd_t \dot{c}}_{=0, \ \text{da $c$ Geodätische}} \right) \dd t \\
&= \langle \covd_s \pdv{c_s}{s}\eval_{s=0}, \dot{c} \rangle \eval^b_a \\
&= 0
\end{align*}
Die letzte Gleichung gilt, da wir Variation mit festen Endpunkten betrachten.
\end{bew}

\begin{defs}[Durchmesser]
Sei $\mfk$ eine zusammenhängende Riemannsche Mannigfaltigkeit.
Dann ist
\begin{align}
    \diam (\mfk):= \sup \{ d(p, q) \vert p, q \in \mfk \} \in (0, \infty]
\end{align}
der Durchmesser von $\mfk$.
\end{defs}

\begin{satz}[Satz von Bonnet-Myers]
\label{satz:bonnetmyers}
Sein $\mfk$ eine vollständige zusammenhängende Riemannsche Mannigfaltigkeit.
Ferner sei $\kappa > 0$ so, dass $\text{ric} \leq \kappa (n - 1) g$ ist 
(Das heißt $\text{ric} (\xi , \eta) \geq \kappa (n - 1) g( \xi \eta)$).
Dann ist $\mfk$ kompakt und es gilt
\begin{align}
    \diam (\mfk) \leq \frac{\pi}{\sqrt{k}}
\end{align}
\end{satz}

\begin{bew}{Beweis von Satz \ref{satz:bonnetmyers}}
Seien $p, q \in \mfk$ mit $\p \neq q$.
Da $\mfk$ Vollständig ist, liefert der Satz von Hopf-Rinow die Existenz einer minimalen
nach Bogenlänge parametrisierten Geodätischen $\gamma: [0 , s] \to \mfk$ 
mit $\gamma (0) =p$ und $\gamma (s) = q$.







\end{bew}<++>

\end{bew}<++>






