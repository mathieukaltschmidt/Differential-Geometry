\section{Satz von Bonnet-Myers}

Unser nächstes Ziel wird es sein den Satz von Bonnet-Myers zu beweisen.
Als Vorbereitung hierfür benötigen wr die zweite Variation der Energie.\\
\textbf{Zur Erinnerung}:
Ist $c_s$ eine glatte Variation von $c: [a, b] \to \mfk$ mit Variationsfeld
$\xi$.
dann ist die erste Variations der Energie, wie folgt gegeben:
\begin{align}
    \dv{s} E(c_s) \eval_{s=0} = - \int^b_a \langle \xi , \covd_t \dot{c} \rangle \dd t
    + \langle \xi , \dot{c} \rangle \eval^b_a
\end{align}

\begin{satz}[Zweite Variation der Energie]
Sei $c: [a, b]\to \mfk$ ein Geodätische und $c_s$ eine Variation von $c$,
mit Variationsfeld $\xi$.
Dann gilt:
\begin{align}
    \dv[2]{s} E(c_s) \eval_{y=0} = \int^b_a  \langle \covd_t \pdv{c_s}{s} , 
    \pdv{c_s}{t} \rangle \dd t
\end{align}

\end{satz}

\begin{bew}
Wir wissen bereits:
\begin{align*}
    \dv{s} E(c_s) \eval_{s=0} = \int^b_a \langle \covd_t \pdv{c_s}{s} , 
    \pdv{c_s}{s} \rangle \dd t
\end{align*}
Damit erhält man:
\begin{align*}
\dv[2]{s}E(c_s)\eval_{s=0}&=\int^b_a \langle\covd_s\covd_t\pdv{c_s}{s},\pdv{c_s}{t}\rangle + 
\langle \covd_t \xi , \covd \pdv{c_s}{t} \eval_{s=0} \rangle \dd t \\
&= \int^b_a \underbrace{\langle\covd_t\covd_s\pdv{c_s}{s}\eval_{s=0},\dot{c}\rangle}_{\ast}
+ \int^b_a \langle \curv (\xi , \dot{c})\xi , \dot{c}
+ \int^b_a \langle\covd_t\xi , \covd_t \xi\rangle\dd t 
\end{align*}

Betrachte $\ast$:
\begin{align*}
\ast &= \int^b_a\left( \pdv{t} \langle \covd_s \pdv{c_s}{s}\eval_{s=0}, c\pdv \right
- \underbrace{\langle\covd_s\pdv{c_s}{s}\eval_{s=0} , 
\covd_t \dot{c}}_{=0, \ \text{da $c$ Geodätische}} \right) \dd t \\
&= \langle \covd_s \pdv{c_s}{s}\eval_{s=0}, \dot{c} \rangle \eval^b_a \\
&= 0
\end{align*}
Die letzte Gleichung gilt, da wir Variation mit festen Endpunkten betrachten.
\end{bew}

\begin{defs}[Durchmesser]
Sei $\mfk$ eine zusammenhängende Riemannsche Mannigfaltigkeit.
Dann ist
\begin{align}
    \diam (\mfk):= \sup \{ d(p, q) \vert p, q \in \mfk \} \in (0, \infty]
\end{align}
der Durchmesser von $\mfk$.
\end{defs}

\begin{satz}[Satz von Bonnet-Myers]
\label{satz:bonnetmyers}
Sein $\mfk$ eine vollständige zusammenhängende Riemannsche Mannigfaltigkeit.
Ferner sei $\kappa > 0$ so, dass $\text{ric} \leq \kappa (n - 1) g$ ist 
(Das heißt $\text{ric} (\xi , \eta) \geq \kappa (n - 1) g( \xi \eta)$).
Dann ist $\mfk$ kompakt und es gilt
\begin{align}
    \diam (\mfk) \leq \frac{\pi}{\sqrt{k}}
\end{align}
\end{satz}

\begin{bew}{Beweis von Satz \ref{satz:bonnetmyers}}
Seien $p, q \in \mfk$ mit $\p \neq q$.
Da $\mfk$ Vollständig ist, liefert der Satz von Hopf-Rinow die Existenz einer minimalen
nach Bogenlänge parametrisierten Geodätischen $\gamma: [0 , s] \to \mfk$ 
mit $\gamma (0) =p$ und $\gamma (s) = q$.
Sei $e \in T_p \mfk$ mit $e \perp \dot{\gamma}(0)$ und $\norm{e}=1$.
Sei $e(t)$ das längs $\gamma$ parallel fortgestzte Vektorfeld.
Setze $\xi (t) := \sin (\frac{\pi}{\delta} t) e(t)$.
Ferner sei $\gamma_s (t)$ eine Variation von $\gamma$ mit festen Endpunkten und
Variationsfeld $\xi (t)$ (z.B. $\gamma_s (t) = \exp (s \xi (t))$).
Da $\gamma$ eine minimale Geodätische ist (nach Konstruktion) gilt:
\begin{align*}
    \dv{s} \eval_{s=0}E(\gamma_s) = 0 \\
\end{align*}
Außerdem gilt:
\begin{align*}
0 \leq & \dv[2]{s}\eval_{s=0} \\
&= \int^\delta_0 \left( \norm{\covd_t \xi}^2 - \langle \curv(\xi, \dot{\gamma}) \dot{\gamma}
, \xi \rangle \right) \dd t\\
&= \int^\delta_0 \left( \norm{ \frac{\pi}{\delta} \cos (\frac{\pi}{\delta} t) e(t) }^2
- \sin^2 (\frac{\pi}{\delta}t) \langle \curv (e, \dot{\gamma})e \rangle \right) \dd t \\
&= \int^\delta_0 \left( \frac{\pi^2}{\delta^2} \cos^2 (\frac{\pi}{\delta} t ) - 
\sin^2 (\frac{\pi}{\delta} t ) k (e, \dot{\gamma}) \right) \dd t
\end{align*}
Wenn $e_1, \dots, e_{n-1}$ eine Orthonormalbasis von $\dot{\gamma} (0)^\perp$ ist, 
so liefert obige Gleichung mit $e= e_j$ nach Summation über $j$:
\begin{align*}
0 &\leq \int^\delta_0 \left( (n-1) \frac{\pi^2}{\delta^2} 
    \cos^2 (\frac{\pi}{\delta} t) - \sin^2 (\frac{\pi}{\delta}t ) 
\text{ric}(\dot{\gamma}, \dot{\gamma}) \right) \dd t \\
&\leq \int^\delta_0 \left( (n-1) \frac{\pi^2}{\delta^2} \cos^2 (\frac{\pi}{\delta}t ) 
- \sin^2 (\frac{\pi}{\delta} ) (n-1)k \right) \dd t \\
&= (n-1) \frac{\pi^2 - k \delta^2}{2\delta}
\end{align*}
Durch umformen erhält man
\begin{align}
    \delta \leq \frac{\pi}{\sqrt{k}} \Rightarrow \diam \mathrm{M} \leq \frac{\pi}{\sqrt{k}}
\end{align}
Da die Mannigfaltigkeit vollständig und beschränkt ist folgt, dass sie kompakt ist.
\end{bew}
\begin{bem}
Aus $k \geq \kappa$ folgt $\text{ric} \geq \kappa (n-1)g$,
daher gilt die Aussage von Bonnet-Myers auch für $k \geq \kappa$.


\section{Jacobi-Felder}

\begin{defs}[Geodätische Variation]
Eine Variation von Kurven 
$c: (-\varepsilon, \varepsilon) \times [a , b] \to \mfk$
heißt geodätische Variation, falls:
\begin{align*}
    t \mapsto c_s(t) := c(s, t)
\end{align*}

eine Geodätische ist für alle $s\in (-\varepsilon, \varepsilon)$.
\end{defs}

Sei $\xi (t) = \pdv{c(0,t)}{s}$ das zugehörige Variationsfeld, dann gilt:

\begin{align*}
\covd^2_t \xi (t) &= \covd_t \covd_t \pdv{s} c(s, t) \eval_{s=0}\\
&= \covd_t \covd_s \pdv{t} c(s, t) \eval_{s=0}\\
&= \underbrace{\covd_s \covd_t \pdv{t} c(s, t) \eval{s=0}}
{=0, \ \text{da Geodätische Variation}} + 
\curv (\pdv{c(0, t)}{t} , \pdv{c(0, t)}{s}) \pdv{c(0,t)}{t}\\
&= \curv ( \dot{c}_0 (t) , \xi (t)) \dot{c}_0 (t)
\end{align*}

\begin{def}[Jacobi-Gleichung]
Die Gleichung
\begin{align}
    \covd^2_t \xi = \curv (\dot{c}_0 , \xi ) \dot{c}_0
\end{align}


heißt \textbf{Jacobi-Gleichung} und Lösungen davon heißen
\textbf{Jacobi-Felder}.
Mit anderen Worten: Variationsfelder von geodätischen Variationen sind Jacobi-Felder.
\end{defs}

\begin{satz}
\label{satz:eindeutigkeitjacobi}
Sei $c_0 : I \to \mfk$ eine Geodätische und $t_0 \in I$.
Dann gilt für alle $\xi, \eta \in T_{c(t_0)}\mfk$ existiert genau ein Jacobi-Feld $J$ 
längs $c$ mit $J(t_0) = \xi$ und $\covd_t J (t_0) = \eta$
\end{satz}

\begin{bew}[Beweis Satz \ref{satz:eindeutigkeitjacobi}]
Sei $e_i (t_0)$ $i\in \{ 1, m \}$ eine Orthonormalbasis von $T_{c(t_0)} \mfk$.
Setze diese mittels Parallelverschiebung längs $c$ zu $e_i(t)$ fort.
Schreibe
\begin{align*}
    J (t) = \sum^n_{j = 1} v^j (t) e_j (t)
\end{align*}
Dann gilt
\begin{align*}
    \covd^2_t J(t) = \sum^n_{j=1} \ddot{v}^j (t) e_j (t)
\end{align*}
Ferner gilt:
\begin{align*}
    \curv( \dot{c}(t), e_j (t)) \dot{c}(t) = \sum^n_{k=1} a^k_j (t) e_k (t)
\end{align*}
Da $J$ Jacobifeld gilt:
\begin{align*}
    \sum^n_{j=1} \ddot{v}^j (t) e_j(t) = \sum^n_{j,k=1}a^k_j v^j e_k \\
    \Rightarrow \ddot{v}^i = \sum_j a^i_j v^j
\end{align*}
Dies ist ein lineares gewöhnliches Differentialgleichungssystem 2. Ordnung.
Daraus folgt es existieren Lösungen, welche eindeutig durch Anfangswert
und Anfangsgeschwindigkeit gegeben sind.
\end{bew}

\begin{bsp}

\begin{itemize}
\item $\mfk$ flach, d.h $\curv = 0$.
Dann gilt:
\begin{align*}
\covd^2_t J = 0
\end{align*}
Jacobifelder sind dann von der Form:
\begin{align*}
J = \xi (t) + t \eta (t)
\end{align*}
wobei $\xi$ und $\eta$ parallel längs $c$ sind.

\item $c$ Geodätische, dann ist $J(t) = (a + bt) \dot{c} (t)$ 
    ein Jacobifeld längs $c$, da $\covd^2_t J(t) = 0$
    \begin{align*}
        \Rightarrow \curv (\dot{c},J) 
        \dot{c} = (a + bt) \curv (\dot{c}, \dot{c})\dot{c} = 0
    \end{align*}
\end{itemize}
\end{bsp}

\begin{lem}
Sei $J : I \to T \mfk$ ein Jacobifeld längs $c$ (Geodätische).
Dann gilt $J(t_0) \perp \dot{c}(t_0)$ und $\covd_t J(t_0) \perp \dot{c}(t_0)$.
Daraus folgt $J(t) \perp \dot{c}(t)$ und $\covd_t J(t)\perp\dot{c}(t)$ $\forall t \in I$
\end{lem}

\begin{bew}




