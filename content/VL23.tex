% !TeX root = ..//diffgeo_main.tex
\section{Flächen}
\begin{defs}
    Sei $n \leq N$ mit $n,N \in \mathbb{N}$.
    $\mfk \subset \R^n$ heißt $n$-dim Untermannigfaltigkeit, falls $\forall p \in \mfk \exists U \subset \mfk$
    offen und $\tilde{U} \subset \R^n$offen, und $\varphi : \tilde{U} \to U$ Homöomorphismus, so dass
\begin{itemize}
    \item $\varphi: \tilde{U} \to \R^n$ glatt
    \item $\covd \varphi \eval_{\tilde{u}}$ ist injekiv $\forall \tilde{u}\in\tilde{U}$
\end{itemize}
Die Abbildung $\varphi \tilde{U} \to U $ heißt Parametrisierung.\\
\textbf{Achtung}: $\varphi^{-1}$ sind Karten und nicht $\varphi$!
\end{defs}
\begin{defs}
Sei $\mfk \subset \R^n$ eine $n$-dim Untermannigfaltigkeit mit $p\in \mfk$.
Dann ist der Tangentialraum definiert als:
\begin{align}
    T_p\mfk = \{ x \in \R^n \vert \exists \ \text{Kurve} \ \gamma : I \to \mfk , \dot{\gamma}(0) = x \} 
\end{align}
Wobei $I \subset \R$ offen ist und $0 \in I$.
\end{defs}

\begin{satz}
\begin{enumerate}
    \item $T_p\mfk$ ist $n$-dim Untervektorraum von $\R^n$
    \item Sei $\varphi : \tilde{U}\to U$ eine lokale Parametrisierung, $p\in\mfk$
        $\tilde{u_0}\in \tilde{U}$, so dass $\varphi (\tilde{u_0})=0$
\end{enumerate}
Dann gilt:\\
$T_p\mfk = \text{Bild} \covd \varphi\vert_{\tilde{u}_0}$ und 
$\pdv{\varphi}{\tilde{u}^1}\eval_{\tilde{u}_0}, \dots , \pdv{\varphi}{\tilde{u}^n} \vert_{\tilde{u}_0}$ bilden Basis von $T_p\mfk$
\end{satz}

\begin{defs}[Erste Fundamentalform]
    $\mfk \subset \R^n$ $n$-dim Untermannigfaltigkeit mit $p\in\mfk$.
    Dann ist die erste Fundamentalform:
    \begin{align}
        g_p : T_p\mfk \times T_p \mfk \to \R
    \end{align}
    definiert durch $g_p (x_p , y_p) = \langle x_p , y_p \rangle_{\R^n}$.
\end{defs}

\begin{bsp}
\begin{itemize}
    \item[a)] Polarkoordinaten auf $\R^2 / \{ 0 \}$ $\varphi (r , \theta) = (r \cos \theta , r \sin \theta)$
    \begin{align*}
        \pdv{\varphi}{r} = ( \cos \theta , \sin \theta) \\
        \pdv{\varphi}{\theta} = (-r \sin \theta , r \cos \theta)
    \end{align*}
    Somit erhält man $g_{ij} = \mqty(1 & 0 \\ 0 & 1) $
\end{itemize}
\end{bsp}
Nun beschäftigen wir uns speziell mit Flächen.
Hierbei interessiert uns besonders die Krümmung von Flächen.
\begin{defs}[Einheitsnormalenvektorfeld]
Sei $\mfk \subset \R^3$ eine $2$-dim Untermannigfaltigkeit (Also eine Fläche)
und $p\in\mfk$.
$N(p)\in\R^3$ heißt Einheitsnormalenvektor zu $\mfk$ in $p$ genau dann, wenn 
$\norm{N(p)}=1$ und $N(p) \perp T_p\mfk$ ist.\\
Das Einheitsnormalenvektorfeld ist die glatte Abbildung $N : \mfk \to \R^3$, so dass $N(p)$ Einheitsvektor für alle $p$.
\end{defs}
\begin{bem}
\begin{itemize}
    \item Es existieren immer genau zwei Einheitsvektorfelder. Wenn eine Parametrisierung $\varphi$ gegeben ist,
        dann sind die Einheitsnormalenvektorfelder wie folgt gegeben:
        \begin{align}
            N(p) = \pm \frac{\pdv{\varphi}{u_1} \times \pdv{\varphi}{u_1}}{\norm{\pdv{\varphi}{u_1} \times \pdv{\varphi}{u_1}}} 
        \end{align}
    \item Es muss nicht immer ein Einheitsnormalenvektorfeld geben (siehe Möbiusband)
\end{itemize}
\end{bem}


\begin{defs}
    $\mfk$ heißt orientierbar, falls ein Einheitsnormalenvektorfeld existiert.
\end{defs}

Die Weingartenabbildung (shape operator) ist ein Maß für die Krümmung.
Sie ist die Änderungsgeschwindigkeit des Tangentialraumes.
Der Tangentialraum wird durch Einheitsvektoren beschrieben.
Mit anderen Worten wir wollen wissen:
wie ändert sich der Einheitsnormalenvektor, wenn man in Richtung eines Tangentialvektors geht.
Sei $X$ die Änderungsrichtung, also $X \in T_p\mfk$.
Das heißt wir sind an $\dd N_p(X)$ insteressiert.

\begin{defs}[Weingartenabbildung]
    Sei $\mfk$ eine orientierbare Fläche mit Einheitsnormalenvektorfeld $N$.
    Dann ist Weingartenabbildung zu $p \in \mfk$:
    \begin{align}
        w_p (X) = - \dd N_p(X)
    \end{align}
\end{defs}

\begin{bem}
    $w_p: T_p\mfk \to T_p\mfk$, denn $\norm{N(p)}=1$. Daraus folgt $N(p)\in S^2 \forall p$.
    Daraus folgt $N: \mfk \to S^2$ und schließlich gilt:
    \begin{align*}
        \dd N_p : T_p\mfk \to T_{N(\varphi)} S^2 = N(p)^\perp = T_p\mfk 
    \end{align*}
\end{bem}

\begin{bem}
    Alternative Definition der Weingartenabbildung:
    Sei $p \in \mfk$ und $X\in T_p\mfk$.
    Wähle eine Kurve $\gamma : I \to \mfk$ mit $\gamma (0) = p$ und $\dot{\gamma}(0) = X$.
    Dann ist 
    \begin{align}
        w_p (X) = - \dv{t} N(\gamma (t))\eval_{t=0}
    \end{align}
    Diese Definition ist unabhängig von der Wahl von $\gamma$.
\end{bem}


\begin{bem}
    Es gilt:
    \begin{align*}
        w (\pdv{\varphi}{u^i} = - \pdv{N}{u^i})
    \end{align*}

    Denn wenn man $\gamma(t) = \varphi (u+te) $ wählt, gilt
    \begin{align*}
        \pdv{t} N(\varphi (u + t e)) = \dv{N}{u^i}
    \end{align*}
\end{bem}

\begin{bsp}
    \begin{itemize}
        \item[a)] Sei $\mfk = xy$Ebene. Dann gilt
        \begin{align*}
            w_p =\smqty(\zmat{2}{2}) 
        \end{align*}
        für alle $p \in \mfk$
    \item Sei $\mfk$ der Zylinder.
        Wir wählen $\varphi (\theta , z) = (\cos (\theta) , \sin (\theta) , 0)$ als Parametrisierung.
        \begin{align*}
            & w(\pdv{\varphi}{z}) = - \pdv{N}{z} = 0 \\
            & w ( \pdv{\varphi}{\theta}) = - \pdv{N}{\theta} = ( \sin ( \theta) , \cos ( \theta) , 0 ) = - \pdv{\varphi}{\theta}
        \end{align*}

        Daraus folgt $w =\mqty(-1 & 0 \\ 0 & 0) $
    \end{itemize}    
\end{bsp}

\begin{satz}
$w_p$ ist selbstadjungiert bezüglich $g_p$, das heißt für alle $p\in\mfk$ gilt:
\begin{align*}
    g_p (X_p , w_p (Y_p)) = g_p (w_p (Y_p) , X_p)
\end{align*}
für alle $X_p, Y_p \in T_p\mfk$
\end{satz}
\begin{bew}
    $g_p$ ist bilinear daher reicht es die Gleichung für eine Basis zu zeigen
    \begin{align*}
        g (X , w Y) &= g ( \pdv{\varphi}{u^i} , w (\pdv{\varphi}{u^j})) \\
        &= \langle \pdv{\varphi}{u} , \underbrace{w ( \pdv{\varphi}{u^j})}_{- \pdv{N}{u^j}} \rangle \\
        &= - \langle \pdv{varphi}{u^i} , \pdv{N}{u^j} \rangle \\
        &= - \langle \frac{\partial^2 \varphi}{\partial u^i \partial u^j} , N \rangle\\
        &= g ( w (\pdv{\varphi}{u^i}) , \pdv{\varphi}{u^j} )
    \end{align*}

    Daraus folgt schließlich:
    \begin{align}
        \langle \pdv{u^j} \pdv{\varphi}{u^i} , N \rangle + \langle \pdv{\varphi}{u^i} , \pdv{N}{u^j} = 0
    \end{align} 
\end{bew}

\subsection{Zweite Fundamentalform}

Wir wollen Krümmung von $\mfk$ mithilfe von Kurven beschreiben die in $\mfk$ verlaufen.
Die Kurven die wir betrachten seien von der Form $ \gamma : I \to \mfk$ mit $\gamma (0) = p$ und $\dot{\gamma} (0)  = X$.\\
\textbf{Betrachte:} $\ddot{\gamma } (0)$\\
\textbf{Problem:} Falls wir $\ddot{\gamma} (0)$ als Maß für die Krümmung von $\mfk$ im Punkt $p$ in Richtung $X$ 
auffassen, dann hängt Krümmung von der Wahl von $\gamma$ ab.\\
Betrachte stattdessen: $\langle \ddot{\gamma} , N(p)$.
\begin{defs}[Normalenkrümmung]
    \begin{align}
     \kappa_N = \langle \ddot{\gamma}(0) , N(p) \rangle
    \end{align}
    heißt Normalenkrümmung von $\gamma$ in $p$.

\end{defs}
\begin{satz}
    \begin{align}
        \kappa_N = g_p (X_p , w_p (X_p))
    \end{align}

    
\end{satz}
\begin{bew}
    \begin{align*}
        \langle \dot{\gamma}(t) , N ( \gamma (t)) \rangle = 0 , \ \forall t \\
        \Rightarrow \langle \ddot{\gamma}(t) , N(\gamma (t))\rangle + \langle \dot{\gamma}(t) , \dv{t}N ( \gamma (t)) \rangle = 0   
    \end{align*}

    Setze $t=0$:
    \begin{align}
        \kappa_N + \langle X_p , - w_p (X_p) \rangle = 0
    \end{align}
\end{bew}

\begin{defs}[Zweite Fundamentalform]
    Die zweite Fundamentalform vin $\mfk$ in $p$ ist:
    \begin{align}
        \pi_p (X_p , Y_p) := g_p (X_p , w_p (Y_p))
    \end{align}
\end{defs}
\begin{bem}
    Die zweite Fundamentalform ist eine Symmetrische Bilinearform.
\end{bem}

\begin{defs}[Hauptkrümmung]
    Die Hauptkrümmung von $\mfk$ in $p$ ist:
    \begin{align}
        & \kappa_1 = \min \pi_p (X , X) \\
        & \kappa_2 = \max \pi_p (X  , X)
    \end{align}

    Dies sind die Eigentwerte der Weingartenabbildung.
    $X$ heißen Hauptkrümmungsrichtungen.
\end{defs}











