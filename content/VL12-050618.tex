% !TeX root = ..//diffgeo_main.tex

Mittels Riemannscher Metriken lassen sich die wesentlichen geometrischen Eigenschaften von Mannigfaltigkeiten beschreiben.
Sie können verwendet werden um Winkel, Distanz bzw. Längen von Kurven zu messen.

\begin{defs}[(Semi-)Riemannsche Mannigfaltigkeit]
Eine differenzierbare Mannigfaltigkeit $\mfk$ ausgestattet mit einer (Semi-)Riemannschen Metrik $g$, wird (Semi-)Riemannsche Mannigfaltigkeit genannt.\\
\textbf{Notation:} $(\mfk, g)$
\end{defs}

Wir wollen nun (Semi-)Riemannsche Metrik $g$ in lokalen Koordinaten betrachten.
Sei $(x, U)$ eine Karte von $\mfk$.
Ferner bezeichnen wir mit $(X_1, \dots, X_n)$ den dazugehörigen lokalen Rahmen von $T\mfk \eval_U$,
das heißt $X_i = \pdv{x_i}$.
Dann sind 
\begin{align}
g_{ij}: &U \to \R \\
& p \mapsto g_{ij} (p) := g\eval_p (X_i (p), X_j (p))
\end{align}

glatte Funktionen und es gilt:
\begin{align}
g_p \left( X\eval_p, Y \eval_p \right) &= g_p \left( \sum_i^m \xi^i X_j \eval_p, \sum_i^m \eta^i X_i \eval_p \right) \\
&= \sum_{ij} \xi^j \eta^i \underbrace{g\eval_p \left( X_j \eval_p, X_i \eval_p\right) }_{g_{ij}(p)}
\end{align}

\begin{defs}[Fundamentalmatrix]
$(g_{ij})^m_{i, j = 1}$ heißt Fundamentalmatrix von $g$ bzgl. $(x, U)$
\end{defs}

\begin{bem}\leavevmode
\begin{enumerate}
\item Matrix $(g_{ij})^m_{i, j = 1}$ ist symmetrisch, d.h. $g_{ij} = g_{ji}$ (Folgt aus Symmetrie von $g_p$)
\item $(g_{ij})^m_{i, j = 1}$ ist invertierbar (Folgt aus der Tatsache, dass $g_p$ nicht entartet ist)
\end{enumerate}
Die Inverse Matrix von$(g_{ij})^m_{i, j = 1}$ bezeichnen wir mit $(g^{ij})^m_{i, j = 1}$
\end{bem}

\begin{bsp}
\begin{itemize}
\item[a)] Sei $V$ ein reeller $n$-dimensionaler Vektorraum und $\langle\cdot,\cdot\rangle$ eine nicht entartete Bilinearform auf $V$.
Für $v, w \in T_p V$ gilt $g_p (v, w) := \langle v, w \rangle$.
Sei $X: V \to V$ glat ($X$ Vektorfeld), dann gilt
\begin{align}
g_p (X(p), Y(p)) = \langle X(p), Y(p) \rangle
\end{align}
ist glatt.
Daraus folgt, dass $g$ eine Semi-Riemannsche Metrik ist.
Es gilt:
\begin{align*}
g_p \left( \pdv{x_i} , \pdv{x_j} \right) &= \Big{\langle} \phi_p \left(\pdv{x_i}\right) , \phi_p \left( \pdv{x_j} \right) \Big\rangle
&= \langle e_i, e_j \rangle = \delta_{ij}
\end{align*}
\item[b)] Sei $\mfk \subset \R^{n-k}$ eine $n$-dimensionale Untermannigfaltigkeit mit
\begin{align*}
&\phi_p: T_p\mfk \to \R^{n+k}\\
& [c] \mapsto \dv{t} \eval_{t=0} c(t)
\end{align*}
wobei $\phi_p$ injektiv und linear ist.
Dann ist 
\begin{align*}
g_p = \phi^\ast_p \langle \cdot , \cdot \rangle 
\end{align*}
positiv definit.
\begin{defs}
Diese Metrik heißt erste Fundamentalform von $\mfk \subset \R^{n+k}$
\end{defs}
\end{itemize}
\end{bsp}

Als nächstes wollen wir die Existenz von Semi-Riemannschen Metriken zeigen.
\begin{satz}
\label{satz:existenz_metrik}
Jede differenzierbare Mannigfaltigkeit kann mit einer Riemannschen Metrik ausgestattet werden.\\
\textbf{Frage:} gilt dies auch für Semi-Riemannsche Metriken? Übung.
\end{satz}

\begin{bew}[Beweis Satz \ref{satz:existenz_metrik}]
Sei $\{ (x_\alpha, U_\alpha) \}$ Atlas von $\mfk$.
Setze
\begin{align*}
(g_\alpha)_{ij} = \delta_{ij}
\end{align*}
Sei $\varphi_\alpha$ eine glatte Partition der Eins mit $\supp \varphi_\alpha \subseteq U_\alpha$.
Dann setze:
\begin{align*}
g := \sum_\alpha \varphi_\alpha g_\alpha.
\end{align*}
$g$ ist hierbei nach Konstruktion glatt.\\
Positivität: Sei $v \in T_p \mfk$, $v \neq 0$, dann gilt:
\begin{align*}
g_p (v, v) = \sum_\alpha \varphi_\alpha (p) \eval{g_\alpha}_p (v, v) \geq 0.
\end{align*}
Es existiert ein $\alpha_0$, so dass $\varphi_{\alpha_0}(p) > 0$.
\begin{align*}
\Rightarrow g_p (v, v) \geq \varphi_{\alpha_0} (p) \eval{g_{\alpha_0}}_p (v, v) > 0.
\end{align*} 
\textbf{Wichtig:} Wir haben Wahlen getroffen. 
$\Rightarrow$ Metrik $g$ ist nicht kanonisch.
Unter Umständen können Metriken sehr verschiedene Eigenschaften haben.
\end{bew}

Seien $\mfk_1$, $\mfk_2$ differenzierbare Mannigfaltigkeiten und unsere Daten sind jetzt $(\mfk_1, g_1)$, $\mfk_2, g_2$.
\begin{align*}
\phi : \mfk_1 \to \mfk_2, \quad \text{Diffeomorphismus}
\end{align*}
Wir wollen zusätzlich, dass $\phi$ die Metriken erhält, das heißt
\begin{align}
\phi^{\ast} g_2 = g_1
\end{align}
Das heißt vom Standpunkt der Riemannschen Geometrie sind $(\mfk_1, g_1)$ und $(\mfk_2, g_2)$ nicht zu unterscheiden,
wenn $\phi: \mfk_1 \to \mfk_2$ ein Diffeomorphismus mit
\begin{align}
\phi^\ast g_2 = g_1
\end{align}
existiert.
Eine solche Abbildung heißt \textbf{Isometrie}.\\

Unsere nächste Frage ist: wie verhält sich $g_{ij}$ unter Kartenwechsel?
\begin{lem}[Transformationsregel]
\label{lem:Transformationsregel}
Seien $(x, U)$ und $(y, V)$ Karten von $p$.
Dann gilt:
\begin{align*}
g^{y}_{ij} &= g(Y_i, Y_j)\\
&= \sum^m_{k,l = 1} \pdv{y^i}(x^k \circ y^{-1}) \pdv{y^j} (x^l \circ y^{-1}) g^x_{kl}
\end{align*}
\end{lem}
\begin{bew}[Beweis Lemma \ref{lem:Transformationsregel}]
Wir wissen:
\begin{align*}
y_i = \sum^m_{k=1} \pdv{y^i} (x^k \circ y^{-1}) \pdv{x^k}
\end{align*}
Daher gilt:
\begin{align*}
g^y_{ij} &= g(Y_i, Y_j)\\
&= \sum^m_{k,l = 1} \pdv{y^i}(x^k \circ y^{-1}) \pdv{y^j} (x^l \circ y^{-1}) \underbrace{g\left( \pdv{x^l}, \pdv{x^k} \right)}_{= g^x_{lk}}
\end{align*}
\end{bew}

Nächstes Ziel: Führe ausgezeichneten Zusammenhang ein.
Dazu benötigen wir zunächst ein Hilfslemma.
Eine Semi-Riemannsche Metrik $g$ definiert einen Isomorphismus zwischen Tangentialbündel und Kotangentialbündel
\begin{align}
&T \mfk \to T^\ast \mfk\\
&T_p \mfk \ni v \mapsto g_p (v, \cdot) \in T^\ast_p \mfk
\end{align}

\begin{hlem}
\label{hlem:formvfkorrespondenz}
Auf einer Semi-Riemannschen Mannigfaltigkeit $(\mfk, g)$ gibt es eine Eins zu Eins Korrespondenz zwischen Vektorfeldern und $1$-Formen
\begin{align}
&\mathfrak{X}(\mfk) \to \Omega^1 (\mfk)\\
& x \mapsto g(X, \cdot)
\end{align}
\end{hlem}

\begin{bew}[Beweis Hilfslemma \ref{hlem:formvfkorrespondenz}]
Wir haben beweis einen Faserweisen Isomorphismus
\begin{align*}
T_p\mfk \to T^\ast_p \mfk
\end{align*}
definiert.
Zu zeigen bleibt:
glattheit, das heißt $X$ glatt, dass ist $g(X, \cdot)$ glatt.
Arbeite in Karte $(x, U)$, dann ist 
\begin{align*}
\left( \eval{\pdv{x_1}}_p, \dots,  \eval{\pdv{x_m}}_p \right)
\end{align*}
eine Basis von $T^\ast_p \mfk$.
Dann gilt $g_p = \sum g_{ij} \dd{x^i} \otimes \dd{x^j}$.
Das heißt:
\begin{align*}
g_p (X, Y) &= g_{ij} \dd{x^i} \otimes \dd{x^j} \left( \sum \xi^k X_k, \eta^l, X_l\right)\\
&= \sum g_{ij} \xi^k \eta^l \delta_{ik} \delta_{kl}\\
&= \sum g_{ij} xi^i \eta^j
\end{align*}
Wir erhalten als $g(X, Y) = \sum g_{ij}$ für $X= \sum \xi^k X_k$ und $Y= \sum \eta^l Y_l$ beliebig.
Wir schreiben nun auch die $1$-Form in lokale Koordinaten:
\begin{align}
\phi = \sum_j \varphi_j \dd{x^j}
\end{align}
Zu Zeigen: sei $\phi = g(X, \cdot)$ dann sind ide Koeffizientenfunktionen glatt.
\begin{align*}
\phi(Z) &= \sum_j \varphi_j \dd{x^j} \left( \sum \eta^l X_l \right)\\
&= \sum_{j} v\varphi_j \eta^l \dd{x^j} (X_l)\\
&= \sum_j \varphi_j \eta^j
\end{align*}
Nach obiger Rechnung haben wir also:
\begin{align}
\varphi_j = \sum^m_{i=1} g_{ij} \xi^i
\end{align}
dies ist eine glatte Funktion.
\end{bew}

\section{Levi-Civita-Zusammenhang}

\begin{satz}
\label{satz:koszulformel}
Sei $(\mfk, g)$ eine Semi-Riemannsche Mannigfaltigkeit, dann existiert auf $\mfk$ ein Zusammenhang mit den folgenden Eigenschaften:
\begin{enumerate}
\item $\covd$ ist Torsionsfrei:
\begin{align}
\covd_Y X - \covd_X Y = \comm{Y}{X}, \quad \forall X, Y \in \mathfrak{X}(\mfk)
\end{align}
\item $\covd$ ist metrisch:
\begin{align}
X g(Y, Z) = g(\covd_X Y, Z) + g(Y, \covd_X Z), quad \forall X,Y,z \in \mathfrak{X}(\mfk)
\end{align}
\end{enumerate}
Dieser Zusammenhang ist eindeutig durch die \textbf{Koszulformel} bestimmt:
\begin{align}
2 g (\covd_X Y, Z) &= X g(Y, Z) + Y g(X, Z) - Z g(X, Y)\\
&\phantom{=} - g(X, \comm{Y}{Z}) - g(Y, \comm{X}{Z}) + g(Z, \comm{X}{Y})
\end{align}
$\covd$ heißt \textbf{Levi-Civita-Zusammenhang}.
\end{satz}

\begin{bew}[Beweis Satz \ref{satz:koszulformel}]
Zu zeigen:
\begin{enumerate}
\item Die beiden geforderten Eigenschaften bestimmen den Zusammenhang und Koszuformel ist erfüllt
\end{enumerate}
Die zweite geforderte Eigenschaft impliziert:
\begin{align*}
&X g(Y, Z) = g(\covd_X Y, Z) + g(Y, \covd_X Z)\\
&Y g(X , Z) = g (\covd_Y X, Z) + g(X, \covd_Y Z)\\
&Z g(X, Y) = g(\covd_Z X, Y) + g(X, \covd_Z Y)
\end{align*}
Daraus folgt:
\begin{align*}
X g(Y, Z) + Y g(X , Z) - Z g(X, Y) &= g(\covd_X Y + \covd_Y X, Z) + g(\covd_X Z - \covd_Z X , Y)\\
 & \phantom{=}+ g(\covd_Y Z - \covd_Z Y, X) \\
 &= 2 g(\covd_X Y, Z) - g(\comm{X}{Y}, Z) + g(\comm{X}{Z}, Y) + g(\comm{Y}{Z}, X)
\end{align*}
Damit ergibt sich
\begin{align}
2 g(\covd_X Y, Z) &= g(\comm{X}{Y}, Z) - g(\comm{X}{Z}, Y) - g(\comm{Y}{Z}, X)\\
&\phantom{=} + X g(Y, Z) + Y g(X, Z) - Z g(X, Y)
\end{align}
Definiere $\covd : \mathfrak{X}(\mfk) \times \mathfrak{X}(\mfk) \to \mathfrak{X}(\mfk)$
durch $\covd (X, Y)$, welches das eindeutig bestimmte glatte Vektorfeld ist, das die Koszulformel erfüllt.\\
Es bleibt zu zeigen:
$\covd : \mathfrak{X}(\mfk) \times \mathfrak{X}(\mfk) \to \mathfrak{X}(\mfk) $ ist ein Zusammenhang

\end{bew}

