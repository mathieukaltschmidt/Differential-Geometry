% !TeX root = ..//diffgeo_main.tex
\begin{satz}
\label{satz:lokalrahmentrivialisierung}
Sei $(E, \pi, \mfk)$ ein Vektorbündel vom Rang $k$.
\begin{enumerate}
\item Aus einem lokalen Rahmen folgt eine lokale Trivialisierung.
Sei $(s_1, \dots, s_k)$ ein lokaler Rahmen über $U\subset\mfk$.
Dann ist 
\begin{align}
\phi : U \times \R^k \to E \eval_U\\
(p, \xi) \to \sum^k_{i=1} \xi_i s_i (p),
\end{align}
eine lokale Trivialisierung
\item  Aus einer lokalen Trivialisierung folgt ein lokaler Rahmen.
Sei $\phi: U \times \R^k \to E \eval_U$ eine lokale Trivialisierung.
Dann ist $(s_1, \dots, s_k)$ ein lokaler Rahmen mit 
\begin{align}
s_i(p) = \phi (p, e_i),
\end{align} 
wobei $\{ e_i \}$ die Standardbasis von $\R^k$ ist.
\end{enumerate}
\end{satz}
\begin{bew}[Teil 1 Satz \ref{satz:lokalrahmentrivialisierung}]
Es gilt, dass 
\begin{align}
\phi \eval_p : \{ p \} \times \R^k \to E \eval_p
\end{align}
ein Isomorphismus ist.
Außerdem hat
\begin{align}
\phi : U \times \R^k \to E \eval_U,
\end{align}
maximalen Rang.\\
Für alle $p$ in $U$ existiert eine Umgebung $V \subset U$ von $p$, so dass die folgende Abbildung eine lokale Trivialisierung ist:
\begin{align}
\psi_V : V\times \R^k \to E \eval_V.
\end{align}
Dann gilt:
\begin{align}
\psi^{-1}_V \circ  \phi (q, \xi) = (q, \underbrace{\psi^{-1}_q \circ \phi_q(\xi)}_{\mathrm{Isomorphismus}})
\end{align}
$\psi^{-1}_V \circ \phi : V \times \R^k \to V \times \R^k$ ist ein Diffeomorphismus.
Daraus folgt, dass $\phi$ maximalen Rang auf $V$ und $U$ hat womit folgt, dass $\phi$ ein Diffeomorphismus ist.
\end{bew}
\begin{bew}[Teil 2 Satz \ref{satz:lokalrahmentrivialisierung}]
Diese Aussage ist sofort klar, da $\phi_p$ ein Isomorphismus ist.
\end{bew}

Lokale Rahmen erlauben es uns mit Schnitten zu rechnen.

\begin{defs}[Hauptteil]
Sei $(s_1, \dots, s_k)$ ein lokaler Rahmen und $\phi$ die dazugehörige lokale Trivialisierung.
Ferner sei $s \in \Gamma_U (E)$ über $U \subset \mfk$.
Dann existiert eine glatte Abbildung
\begin{align}
\sigma : U \to \R^k,
\end{align}
so dass
\begin{align}
&s(p) = \sum^{k}_{i=1} \sigma_i (p) s_i(p)\\
&\phi(p, \sigma(p)) = s(p).
\end{align}
$\sigma$ heißt der Hauptteil von $s$ bezüglich $\phi$.
\end{defs}

\begin{bem}
Die Aussagen: $\sigma$ ist glatt und $s$ ist glatt sind äquivalent.\\
Sei $(t_1, \dots t_k)$ ein lokaler Rahmen über $V$ und $\psi$ die dazugehörige lokale Trivialisierung, so dass $U \cap V \neq \emptyset$.
Über $U \cap V$ gilt:
\begin{align}
s_i = \sum_j g^j_i t_j,
\end{align}
wobei $g^j_i : U \cap V \to \R$.
Setze $g(p) = (g^j_i (p))^k_{i,j =1}$
\begin{align}
&g(p)(t_1(p), \dots , t_k(p)) = (s_1(p), \dots, s_k(p))\\
&g: U \cap V \ni p \to g(p) \in \mathrm{GL}(E\eval_p)
\end{align}
sei $s \in \Gamma_{U \cap V} (E)$ und Hauptteile $\sigma_\phi$, $\sigma_\psi$, dann ist
\begin{align}
&\sigma^i_\phi = \sum^k_{j=1} g^j_i \sigma^j_{\psi}\\
&\sigma_\phi = g \sigma_\psi\\
&g : U \cap V \to \mathrm{GL} (k, \R)
\end{align}
\end{bem}


\begin{defs}[Pullback]
Sei $E	\xrightarrow{\pi} \mfk$ ein Vektorbündel und $f : \mfka \to \mfk$ eine glatte Abbildung.
Der Pullback von $E$ über $f$ ist das Vektorbündel $f^* E$ welches definiert ist durch:
\begin{enumerate}
\item $(f^* E)_{p \in \mfka} = \{ (p, x) \vert x \in E_{f(p)}\}$
\item sei $\phi : U \times \R^k \to E\eval_U$ lokale Trivialisierung von $E$
\begin{align}
&f^* \phi : f^{-1} (U) \times \R^k \to (f^* E)\eval_{f^{-1}(U)}\\
&(p, \xi) \mapsto (p, \phi(f(p), \xi)) 
\end{align}
\end{enumerate}
\end{defs}


\begin{defs}
Ein Schnitt von $E$ entlang von $f$ ist eine glatte Abbildung $\delta : \mfka \to E$, so dass $\pi \circ s = f$. 
\end{defs}

\section{Zusammenhang und kovariante Ableitung}

\begin{defs}[Lie-Klammern]
\begin{align}
&\comm{\cdot}{\cdot} : \mathfrak{X}(\mfk) \times \mathfrak{X}(\mfk) \to \mathfrak{X}(\mfk)\\
&\comm{X}{Y} f := X(Y(f)) - Y(X(f))
\end{align}
Hier bleibt als Übung zu zeigen, dass $\comm{X}{Y}$ tatsächlich ein neues Vektorfeld ist.\\
Zusammen mit der Lie-Klammer ist $\mathfrak{X}(\mfk)$ eine Lie-Algebra.
\end{defs}

\begin{defs}[Zusammenhang]
Sei $(E, \pi, \mfk)$ ein Vektorbündel vom Rang $k$.
Ein Zusammenhang auf $E$ ist eine Abbildung
\begin{align}
&\covd : \mathfrak{X}(\mfk) \times \Gamma(E) \to \Gamma (E) \\
&(X, s) \mapsto \covd (X, s) = \covd_X s
\end{align}
Wobei folgende Eigenschaften erfüllt sind:
\begin{enumerate}
\item $\covd$ ist tensoriell in $X$: 
\begin{align}
&\covd_{X_1 + X_2} s = \covd_{X_1} s + \covd_{X_2} s \\
&\covd_{\phi X} s = \phi \ \covd_X s
\end{align}
\item $\covd$ ist eine Derivation in s:
\begin{align}
&\covd_X (s_1 + s_2) = \covd_X s_1 + \covd_X s_2\\
&\covd_X (\phi s) = X (\phi) s + \phi \ \covd_X  s
\end{align}
\end{enumerate}
\end{defs}

Wir führen hier die folgende Notation ein:
$\covd_X s$ heißt die kovariante Ableitung von $s$ in Richtung $X$.\\
\textbf{Wichtiger Spezialfall}:
\begin{align}
&E = T\mfk\\
&\covd: \underbrace{\mathfrak{X}(\mfk)}_{\mathrm{tensoriell}} \times \underbrace{\mathfrak{X}(\mfk)}_{\mathrm{derivativ}} \to \mathfrak{X}(\mfk)
\end{align}

\begin{bsp}
Sei $E = \mfk \times \R^k$ das triviale Bündel mit
\begin{align}
&s: \mfk \to E \\
&p \mapsto (p, \sigma(p))
\end{align}
wobei $\sigma = (\sigma_1, \dots, \sigma_k)$, $\sigma_i \in \mathcal{F}(\mfk)$.
Dann ist der kanonische Zusammenhang gegeben als:
\begin{align}
(\covd_X s)(p) = (p, X_p(\sigma_1), \dots, X_p(\sigma_k))
\end{align}
Wir benutzen die folgende Notation: $\covd_X s = X(\sigma)$.
\end{bsp}

\begin{lem}
\label{lem:lokalisierung1}
$X_1, X_2 \in \mathfrak{X}(\mfk)$ und $X_1(p) = X_2(p)$,
dann folgt daraus, dass 
\begin{align}
(\covd_{X_1} s)(p) = (\covd_{X_2} s) (p).
\end{align}

\end{lem}

\begin{lem}
\label{lem:lokalisierung2}
$s_1, s_2 \in \Gamma(\mfk)$ und $s_1 = s_2$ in einer Umgebung von $p$,
daraus folgt, dass
\begin{align}
(\covd_{X} s_1)(p) = (\covd_{X} s_2) (p).
\end{align}
\end{lem}

\begin{bew}[Lemma \ref{lem:lokalisierung2}]
Wähle $\phi \in \mathcal{F}(\mfk)$ mit $\supp \phi \subseteq U$ und $\phi = 1$ auf einer Umgebung $V \subset U$ von p.
Dann gilt 
\begin{align}
&\phi s_1 = \phi s_2 \\
&\covd_X(\phi s_1)(p) = \covd_X(\phi s_2)(p)
\end{align}
Für die linke Seite ist
\begin{align}
\covd_X(\phi s_1)(p) = \underbrace{X(p)}_{=0} s_1(p) + \underbrace{\phi(p)}_{=1} \covd_X s_1(p) = \covd_X s_1 (p).
\end{align}
Das gleiche gilt für die rechte Seite und somit folgt die Aussage:
\begin{align}
(\covd_{X} s_1)(p) = (\covd_{X} s_2) (p).
\end{align}
\end{bew}
