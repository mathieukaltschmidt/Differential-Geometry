\begin{satz}[Umkehrsatz]
Sei $f: M \rightarrow N$ glatt, $p\in M$. Sei $\d f\vert_p: T_p M\rightarrow T_{f(p)} N$ ein Isomorphismus. Dann existiert eine Umgebung $U$ von $p$ und $U'$ von $f(p)$, so dass
\begin{align*}
f\vert_U: U \rightarrow U'
\end{align*}
ein Diffeomorphismus ist.
\end{satz}

\begin{bew}
Nutze Karten um dies auf den euklidischen Fall zurückzuführen
\missingfigure{Karten und Abbildungen}
Seien $(x, U)$ und $(y, U')$ Karten von $M$ und $N$ um $p$ und $f(p)$.\\
o. B. d. A.: $f(U) \subset U'$\\
Dann ist $\varphi$ eine glatte Abbildung, deren Differential $\d f\vert_{x(p)}$ invertierbar ist.\\ 
Umkehrsatz im $\R^n$ liefert: $\exists$ Umgebungen  $\hat{V}$ von $x(p)$ und $\hat{V}'$ von $y(f(p))$, so dass $\varphi\vert_{\hat{V}'}$ ein Diffeomorphismus ist. Dann ist $f\vert_{x^{-1}(\hat{V})}: x^{-1}(\hat{V}) \rightarrow y^{-1}(\hat{V}')$ ein Diffeomorphismus.
\end{bew}

\begin{satz}[implizite Funktion]
$f:M^m \rightarrow N^m$ glatt.
\begin{enumerate}
\item Sei $\operatorname{Rang} f = r$. Dann $\exists$ zu jeder Karte $(y, U')$ um $f(p)$ eine Karte $(x, U)$ um $p$, so dass\linebreak \mbox{$y\circ f\circ x^{-1}(u^1, \dots, u^n) = (u^1, \dots, u^r, \varphi^{r+1}(u), \dots, \varphi^n(u))$}.
Falls $y(f(p)) = 0$, so kann man $x$ so wählen, dass $x(p) = 0$ und $\varphi^j(0) = 0$ und $(\partial_i \varphi^j)(0) = 0\ \forall i, j > r$.
\item Sei $\operatorname{Rang	}f = r (\forall p \in M!)$. Dann gibt es Karten $(x, U), (y, U')$, so dass\linebreak \mbox{$y\circ f\circ x^{-1}(u^1, \dots, u^n) = (u^1, \dots, u^r, 0, \dots, 0)$}.
\end{enumerate}
\end{satz}

\begin{bew}[zu 1]
"Wähle Karten und modifiziere diese geschickt"\\
Sei $(y, U')$ Karte von $N$ um $f(p)$ und $(x, U)$ Karte von $M$ um $p$ mit $\hat{x}(p) = 0$. Setze 
\begin{align*}
\hat{\varphi} &= y\circ f\circ \hat{x}^{-1}\\
\hat{A} &= (A_{ij} = (\partial_i \varphi^j)_{\substack{1\leq i\leq m\\1\leq j\leq n}}
\end{align*}
Da $\operatorname{Rang}f = r$ können wir o. B. d. A annehmen, dass $\det(\tilde{A})\not = 0$, wobei\mbox{$\tilde{A} = (\hat{A})_{1\leq i, j\leq r}$}\\
Setze $x^i =
\left\{
\begin{array}{l}
y'\circ f, 1\leq i\leq r\\
\hat{x}^i, r+1\leq i\leq n
\end{array}
\right.$
Dann gilt $c(p) = 0$ und
\begin{align*}
\partial_i(x^j\circ\hat{x}^{-1})(0) = 
\left(
\begin{array}{c|c}
\partial_i\hat{\varphi}^j(0) & \ast\\
\hline
0 & \operatorname{id}
\end{array}
\right)
\end{align*}
$\Rightarrow x$ hat $\operatorname{Rang} = m = \dim M$ im Punkt $p$\\
$\Rightarrow x$ lokaler Diffeomorphismus\\
$\Rightarrow \exists$ Umgebung $U$ um $p$ und $V$ von $0$ in $\R^n$, so dass $x: M \rightarrow V$ eine Karte von $M$ ist und 
\begin{align*}
varphi(u^1, \dots, u^n) &= y\circ f\circ x^{-1}(u^1, \dots, u^n)
&= (u^1, \dots, u^r, \varphi^{r + 1}(u), \dots, \varphi^n(u))
\end{align*}
wobei $\varphi^\alpha$ glatt auf $U'$ sind, mit $\varphi'(0) = 0$.
Betrachte Jacobimatrix:
\begin{align*}
A_{ij} &= (\partial_i \varphi^j)
&= 
\begin{pmatrix}
\mathds{1} & 0\\
\ast & (\partial_i\varphi^j)
\end{pmatrix}
\end{align*}
Da $\operatorname{Rang}\varphi = r$ in einer Umgebung von $u=0$ hat $\Rightarrow \partial_i\varphi^j = 0\ \forall i, j > r$
\end{bew}

\begin{kor}
Sei $f: M^m \rightarrow N^n$ glatt.
\begin{enumerate}
\item Sei $q\in N$ regulärer Wert, so ist
\begin{align*}
H\coloneqq f^{-1}(q) = \set{p\in M\middle| f(p) = q}
\end{align*}
eine Untermannigfaltigkeit der Dimension $m-n$
\item Sei $f$ in einer Umgebung von $H = f^{-1}(q)$ vom Rang $r$, so ist $H$ eine Untermannigfaltigkeit der Dimension $m-r$
\end{enumerate}
Der Tangentialraum $T_p H$ ist isomorph zu $\ker \d f\vert_p \subseteq T_p M\ \forall p \in H$.
\end{kor}