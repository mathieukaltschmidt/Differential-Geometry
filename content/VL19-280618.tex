\chapter{Krümmung 2.0}
\section{Schnittkrümmung}
Zur Erinnerung:
\begin{align}
	\curv(X,Y)Z = \covd_X(\covd_YZ)-\covd_Y(\covd_XZ)-\covd_{[X,Y]}Z
\end{align}
wobei $\covd$ ein gegebener Zusammenhang auf $\mfk$ ist. \\
Wir wollen nun Riemannsche Mannigfaltigkeiten betrachten. Sei $(M,g)$ gegeben. Das bedeutet es existiert ein Levi-Civita-Zusammenhang auf $\mfk$. \\
Auf einer Riemannschen Mannigfaltigkeit hat der Krümmungstensor mehr Symmetrien und wir können einfacher geometrische Krümmungsbegriffe definieren.
\begin{satz}
Seien $X, Y, U$ und $V \in $ V($\mfk$).
Dann gelten die folgenden Aussagen:	
\begin{enumerate}
	\item $\curv(X,Y)Z = -\curv(Y,X)Z$
	\item $\curv(X,Y)Z +\curv(Y,Z)X+\curv(Z,X)Y = 0$
	\item $g(\curv(X,Y)U,V)=-g(\curv(X,Y)V,U)$
	\item $g(\curv(X,Y)U,V)=g(\curv(U,V)X,Y)$
\end{enumerate}
\end{satz}
\begin{bew}
Wir beginnen zunächst einmal mit dem Beweis der dritten Aussage:
\begin{align*}	
g(\curv(X,Y)U,V)+g(\curv(X,Y)V,U) \overset{!}{=} 0
\end{align*}
%TODO Hier fehlt noch der Rest vom Beweis..
\end{bew}

\begin{defs}
Nachfolgend treffen wir die Konvention:
	\begin{align}
	\curv(X,Y,U,V) = g(\curv(X,Y)U,V)	
	\end{align}
\end{defs}
Wir setzen nun für alle $X,Y \in T_p\mfk$:
\begin{align}
	k(X,Y) =\curv(X,Y,Y,X)
\end{align} 
\begin{satz}
	Der Krümmungstensor $\curv$ ist eindeutig durch $k$ bestimmt.
\end{satz}
\begin{bew}
Zunächst einmal betrachten wir den Ausdruck
\begin{align}
3\curv(X,Y)Z &= \curv(X,Y+Z)(Y+Z) - \curv(X,Y)Y -\curv(X,Z)Z \nonumber \\ 
&-\curv(Y,X+Z)(X+Z) + \curv(Y,X)X-\curv(Y,Z)Z \tag{$\dagger$}
\end{align}
Damit ergibt sich für den zweiten Teil:
\begin{align}
	2g(\curv(X,Y)Y,Z) = k(X+Z,Y) - k(X,Y) - k(Z,Y)
\end{align}
Es folgt insgesamt mit Hilfe der ersten und zweiten Gleichung:
\begin{align}
	6g(\curv(X,Y)Z,W) = 2g(\dagger,W) = \text{18 Terme in }k
\end{align}
\end{bew}
\textbf{Problem:} $k$ hängt von der Wahl von $X$ und $Y$ ab. Wir wollen ein Objekt definieren, welches lediglich von der von $X$ und $Y$ aufgespannten Ebene abhängt. \\
Sei $\tilde{X} = aX +bY$ sowie $\tilde{Y} = cX +dY$. Dann folgt:
\begin{align}
	k(\tilde{X},\tilde{Y}) &= g(\curv(\tilde{X},\tilde{Y})\tilde{Y},\tilde{X}) \nonumber \\
	&= (ad-bc)^2 g(\curv(X,Y)Y,X) \nonumber\\
	&= (ad-bc)^2 k(X,Y)
\end{align}
Wir müssen $k$ mit einer Größe skalieren, welche sich auf die gleiche Weise transformiert:
\begin{align}
	k_1(X,Y) = g(X,X)g(Y,Y) - g(X,Y)^2
\end{align}
Denn dann gilt nämlich:
\begin{align}
	k_1(\tilde{X},\tilde{Y}) = (ad-bc)^2 k_1(X,Y)
\end{align}
\begin{defs}[Schnittkrümmung]
Sei $\sigma \subset T_p\mfk$ eine Ebene. Die Schnittkrümmung von $\nabla$ ist wie folgt definiert:
\begin{align}
	\mathcal{K}(\sigma) = \frac{k(X,Y)}{k_1(X,Y)} = \frac{g(\curv(X,Y)Y,X)}{g(X,X)g(Y,Y)-g(X,Y)^2} 
\end{align}	
für alle Basen $(X,Y)$ von $\sigma$.
\end{defs}
\begin{satz}
Die Schnittkrümmungen von allen Ebenen in $T_p\mfk$ bestimmen $\curv_p$. Analog liefert uns die Aussage, dass für bekanntes $\curv$	alle $\mathcal{K}$ bekannt sind und umgekehrt.
\end{satz}
	
\section{Ricci-Krümmung}
Die Ricci-Krümmung ist eine wichtige "Vereinfachung" des Krümmungstensors. \\
Zunächst, als Vorbereitung, definieren wir den \textbf{Ricci-Tensor:}
\begin{align}
	\operatorname{ric}_p(Y,Z) := \operatorname{tr}(\curv(-Y)Z) \quad \in \R
\end{align}
das heißt bezüglich einer Orthonormalbasis von $T_p\mfk$, welche wir mit $E_i$ bezeichenen gilt: 
\begin{align}
	\operatorname{ric}_p(Y,Z) = \sum_{i=1}^{n} g_p(\curv(E_i,Y)Z,E_i)
\end{align}
\begin{defs}[Ricci-Krümmung]
	Für einen Vektor $Y \in T_p\mfk$ ist die \textbf{Ricci-Krümmung in Richtung Y} definiert als:
	\begin{align}
		\operatorname{Ricci}(Y) = \frac{\operatorname{ric}_p(Y,Y)}{g_p(Y,Y)}
	\end{align} 
\end{defs}
Durch Anwendung der Bianchi-Identität erhalten wir die nützliche Eigenschaft: 
\begin{align}
	\operatorname{ric}(X,Y) = \operatorname{ric}(Y,X)
\end{align}
\section{Skalarkrümmung}
Die Kenntnis des Riemann-Tensors $\curv$ liefert 
wie bereits zuvor erwähnt alle Schnittkrümmungen $\mathcal{K}$ und umgekehrt. Aus den $\mathcal{K}$ erhalten wir dann die Ricci-Krümmung und schließlich die Skalarkrümmung $\operatorname{Scal}$. \\
Diese ist eine Abbildung $\mfk \rightarrow \R$, welche wir wie folgt definieren:
\begin{align}
	\operatorname{Scal}(p) = s(p) = \operatorname{tr}(\operatorname{Ricci_p}) 
\end{align} 
Also erhalten wir bezüglich einer Orthonormalbasis $E_i$ von $T_p\mfk$:
\begin{align}
	\operatorname{Scal}(p) &= \sum_{i,j = 1}^{n}g(\curv(E_i,E_j)E_j,E_i) \nonumber \\
	&= \sum_{i}\operatorname{Ricci}(E_i)
\end{align}