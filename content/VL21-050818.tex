\begin{lem}

Sei $\Jac_\gamma \mfk$ der Vektorraum der Jacobifelder entlang $\gamma$, wobei 
$\mfk$ eine $n$-dimensionale Riemannsche Mannigfaltigkeit ist.
Dann gilt:
\begin{itemize}
\item[a)] $\dim (\Jac_\gamma \mfk) = 2n$
\item[b)] $\{ J \in \Jac_\gamma \mfk \vert J \perp \gamma' \}$ ist 
    $(2n-2)$-dimensionaler Vektorraum
\item[c)] $\{ J \in \Jac_\gamma \mfk \vert J \perp \gamma' , J(0)=0\}$ ist 
    $(n-1)$-dimensionaler Vektorraum
\end{itemize}
Ferner gilt: $\langle J(t) , \gamma'(t) = at + b$ mit $a,b \in \R$.
\end{lem}
\begin{bew}
$J$ ist eindeutig durch Anfangsbedingungen $(J(0),J'(0))\in\R^{2n}$ bestimmt.\\
\textbf{Behauptung}: $f: \R\to\R; t \mapsto \langle J(t) , \gamma'(t)\rangle$
ist linear.\\
\textbf{Beweis der Behauptung}:
\begin{align*}
\dv{t}\langle J(t), \gamma'(t)\rangle &= \langle \covd_t J(t), \gamma'(t) \rangle 
+ \langle J(t),\underbrace{\covd_t \gamma'(t)}_{=0}\\
&=  \langle J'(t) , \gamma'(t)\rangle
\end{align*}
Wobei wir $J'(t):=\covd_t J(t)$ definieren.
\begin{align*}
\dv[2]{t} \langle J(t), \gamma'(t)\rangle &= \langle J''(t), \gamma'(t)\rangle \\
&= -\langle\curv(J, \gamma')\gamma', \gamma'\rangle\\
&= 0
\end{align*}

In der vorletzten Gleichung wurde die Jacobi-Gleichung verwendet.
Daraus folgt, dass $\langle J(t), \gamma'(t) \rangle = at + b$ gilt,
wobei $b=\langle J(0), \gamma'(0) \rangle$ und $a = \langle J'(0), \gamma'(0)\rangle$.\\
\textbf{zu b)}\\
Aus $\langle J, \gamma' \rangle = 0 $ folgt $a = b = 0$.
Das heißt die Bedingung beschreibt einen Untervektorraum der Kodimension $2$ im $2n$-dimensionalen
Raum.\\
\textbf{zu c)}\\
Wenn zusätzlich $J(0)=0$ gilt folgt $b = 0$.
Daher hat man einen Untervektorraum der Kodimension $1$ im $n$-dimensionalen Raum
der Anfangsbedingung $J'(0)$.
\end{bew}


\begin{lem}
    Sei $\gamma$ eine Geodätische $\gamma : [0,a]\to\mfk$.
    Sei außerdem $J$ ein Jacobifeld entlang $\gamma$ mit 
    $J(0)=0$, $J'(0)=w \in T_p\mfk$.
    Dann gilt:
    \begin{align*}
        J(t) = t (d \exp_p)_{tv} (w)
    \end{align*}
\end{lem}

\begin{satz}
    Sei $\gamma : I \to \mfk$ eine Geodätische.
    Sei $\xi$ ein glattes Vektorfeld längs $\gamma$.
    Dann gilt\\
    $\xi$ ist ein Jacobifeld $\leftrightarrow$ $\xi$ ist Variationsfeld einer Geodätischen Variation.
\end{satz}









