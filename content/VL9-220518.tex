% !TeX root = ..//diffgeo_main.tex

Unser nächster Schritt ist es nun die Parallelverschiebung zu definieren.
Auf dem Weg dorthin ist unser erstes Ziel zunächst die kovariante Ableitung von Schnitten längs einer Abbildung zu definieren.
\begin{enumerate}
\item Zu Schnitten längs einer Abbildung:
\begin{align}
\begin{xy}
  \xymatrix@=0.2\linewidth{
          &   E \ar[d]_{\pi} \\
      \mfka \ar[ur]^{s^\mfka} \ar[r]_f  &   \mfk
  }
\end{xy}
\end{align}
Eine Abbildung $s: \mfka \to E$ heißt Schnitt längs $f: \mfka \to \mfk$ falls:
\begin{enumerate}[label=\roman*]
  \item $s^\mfka$ ist glatt
  \item Das obere Diagramm kommutiert: $f = \pi \circ s^\mfka$
\end{enumerate}
Notation: $\Gamma_f (E)$\\
Wichter Spezialfall: $\mfka = I$, das heißt $f$ ist eine Kurve falls $E = T\mfk$
\begin{align}
\begin{xy}
  \xymatrix@=0.2\linewidth{
          &   T\mfk \ar[d]_{\pi} \\
      I \ar[ur]^{s^I} \ar[r]_f  &   \mfk
  }
\end{xy}
\end{align}
\item Wollen $\covd: \mathfrak{X}(\mfka) \times \Gamma_f (E) \to \Gamma_f (E)$.
\end{enumerate}
\begin{defs}[Kovariante Ableitung längs eines Schnittes]
Sei $\phi = (s_1, \dots, s_k)$ ein lokaler Rahmen von $E$ über $U$ und $s \in \Gamma_f (U)$. 
Dann ist:
\begin{align}
s = \sum^{k}_{i=1} \sigma_i (s_i \circ f)
\end{align}
Dann definieren wir die kovariante Ableitung längs $f$ wie folgt:
\begin{align}
\covd^{f}_x s &= \sum^k_{j=1} \left( x(\sigma_j) + \sum^k_{i=1} \omega_{ij}(f_\ast x)\sigma_i \right) s_j \circ f\\
&= x(\sigma) + (f^\ast \omega)(x) \sigma
\end{align}
Die Wohldefinertheit soll als Übung gezeigt werden.
\end{defs}
\begin{satz}
Die kovariante Ableitung $\covd^f$ längs $f$
\begin{align}
\covd^f: \mathfrak{X}(\mfka) \times \Gamma_f (E) \to \Gamma_f (E),
\end{align}
ist tensoriell im ersten Argument und derivativ im zweiten Argument.
\end{align}
\end{satz}
Wenn wir die Schnitte $\Gamma_f (E)$ mit $\Gamma(f^\ast E)$ identifizieren, dann erhalten wir den zurückgezogenen Zusammenhang
\begin{align}
f^\ast \covd : \mathfrak{X}(\mfka) \to \Gamma(f^\ast E) \to \Gamma(f^\ast E).
\end{align}

\begin{satz}
Sei $s^\mfk \in \Gamma(E)$, $q \in \mfka$ und $v\in T_q \mfka$.
Dann gilt:
\begin{align}
\covd^f_v (s\circ f) = \covd_{f^\ast v} s
\end{align}
Die wichtigste Situation ist hierbei die folgende:
\begin{align}
\begin{xy}
  \xymatrix@=0.2\linewidth{
          &   T\mfk \ar[d]_{\pi} \\
      c : I  \ar[r]_f  &   \mfk
  }
\end{xy}
\end{align}
Hier gilt
\begin{align}
\covd_t s := \covd^c_{\pdv{t}} s
\end{align}
\end{satz}

\begin{bem}
Sei $c$ so gewählt, dass $\dot{c}(t) = 0$, das heißt $c(x)=p$.\\
\textbf{Achtung:} $\covd_t s$ kann ungleich Null sein, selbst wenn $\dot{c}(t) = 0$.
Zum Beispiel: $c(t)=0$ und $s \in \Gamma_c (E)$, dann muss $s(t)\in E_p$ nicht konstant sein.
Dann ist $\covd_t s = \pdv_t s $ Ableitung um $s(1)$ als Abbildung in $E_p \cong \R^k$.
\end{bem}
Als nächstes wollen wir nun die kovariante Ableitung längs von Kurven verwenden, um die Parallelverschiebung zu definieren.
\begin{defs}[Parallelität]
Sei $c: I \to \mfk$ eine glatte Kurve und $(\pi, E, /mfk)$ ein Vektorbündel.
Ein Schnitt $s^I \in \Gamma_c (E)$ heißt parallel längs $c$ falls 
\begin{align}
\covd^c_t s = 0.
\end{align}
\end{defs}
\begin{bem}
Wenn $s_1$ und $s_2$ parallel sind, so sind auf Linearkombinationen $\alpha s_1 + \beta s_1$ ($\forall \alpha, \beta \in \R$) parallel.
\end{bem}
In einem lokalen Rahmen bedeutet $\covd^c_t s = 0$ folgendes:
\begin{align}
\dot{\sigma} + \omega(\dot{c}) \sigma = 0.
\end{align}
Dies ist eine gewöhnliche Differentialgleichung erster Ordnung.

\begin{lem}
Sei $t_0 \in I$ und $x \in E_{c(t_0)}$ dann existiert genau ein paralleler Schnitt $s$ längs $c$ mit $s(t_0 = x)$

\end{lem}