\chapter{Krümmung}

\begin{defs}[Krümmungstensor]
Der Krümmungstensor eines Zusammenhangs $\operatorname{D}$ auf $E$ ist die Abbildung:
\begin{align*}
\curv: \mathfrak{X}(\mfk) \times \mathfrak{X}(\mfk) \times \Gamma(E) \quad &\rightarrow \quad \Gamma(E) \\
(X, Y, s) \quad &\mapsto \quad \curv(X,Y)s \\
\end{align*}
Hierbei gilt:  \quad $\curv(X,Y)s = \covd_X(\covd_Ys) - \covd_Y(\covd_Xs) - \covd_{[X,Y]}s$
\end{defs}

\textbf{Achtung:} Die Vorzeichen-Konvention ist in der Literatur im Allgemeinen nicht eindeutig!

\begin{satz}
Der Krümmungstensor ist tensoriell in $X, Y$ \& $s$. Außerdem ist er schiefsYmmetrisch in $X$ \& $Y$. 
\begin{bew}
\begin{itemize}
\item $\curv(X,Y) = -\curv(Y,X)$ trivial $\Rightarrow$ SchiefsYmmetrie klar.
\item zu zeigen: $\curv$ ist tensoriell in $s$, das heißt $\curv(X,Y)(\varphi s) = \varphi \curv(X,Y)s$

	\begin{itemize}
		\item[a)] $\covd_X(\covd_Y(\varphi s)) = \covd_X(Y(\varphi)s + \varphi\covd_Ys) = X(Y(\varphi))s + Y(\varphi)\covd_Xs + X(\varphi)\covd_Ys  + \varphi\covd_X(\covd_Ys)$
		\item[b)] $-\covd_Y(\covd_X(\varphi s)) = -Y(X(\varphi))s + -X(\varphi)\covd_Ys - Y(\varphi)\covd_Xs  + \varphi\covd_Y(\covd_Xs)$
		\item [c)] $-\covd_{[X,Y]}(\varphi s) = -[X,Y](\varphi)s - \varphi\covd_{[X,Y]}s$
	\end{itemize}
\end{itemize}
\vspace{.5cm}
$\Rightarrow \curv(X,Y) \varphi s = \varphi \covd_X(\covd_Ys) - \varphi \covd_Y(\covd_Xs) - \varphi \covd_{[X,Y]}s 
= \varphi \curv(X,Y)s $
\end{bew}
\begin{itemize}
\item  zu zeigen: $R$ ist tensoriell in X.
\end{itemize}
\begin{hlem}
$[\varphi X,Y] = \varphi[X,Y] - Y(\varphi) X$
\begin{bew}
$[\varphi X,Y]f = \varphi X(Y(f)) - Y(\varphi X(f)) = \varphi X(Y(f)) - Y(\varphi) X(f) - \varphi Y(X(f))$
\end{bew}
\end{hlem}
\begin{bew}
\begin{align*}
\curv(\varphi X, Y) s &= \covd_{\varphi X} (\covd_Ys) - \covd_Y(\covd_{\varphi X} s) - \covd_{[\varphi X,Y]}s \\
&= \varphi \covd_X (\covd_Ys) - Y(\varphi) \covd_Xs - \varphi \covd_Y(\covd_Xs) - \covd_{\varphi [X,Y]}s + \covd_{Y(\varphi)X}s \\
&= \varphi \curv(X,Y)s
\end{align*}
\end{bew}
\end{satz}
\begin{kor}
$\curv_p: T_p\mfk \times T_p\mfk \times E_p  \rightarrow E_p$ ist punktweise definiert $\forall p \in \mfk$
\end{kor}
\begin{bem}
In vielen Büchern wird zunächst der Fall $E= T\mfk$ diskutiert.
\end{bem}

\begin{bsp}
1) \ Kanonischer Zusammenhang auf $\mfk \times \R^k:$
\begin{align*}
s \in \Gamma(E) \quad \longleftrightarrow \quad \sigma: \mfk \rightarrow \R^k 
\end{align*}
In diesen Fall ist der Krümmungstensor wie folgt gegeben:
\begin{align*}
R(X,Y)s &= \covd_X(\covd_Ys) - \covd_Y(\covd_Xs) - \covd_{[X,Y]}s \\
			&= X(Y(\sigma)) - Y(X(\sigma)) - [X,Y]\sigma \\
			&= 0
\end{align*}
\end{bsp}

\begin{defs}
\begin{itemize}
	\item[a)] Ein Zusammenhang $\covd$ auf $E$ mit $R \equiv 0$ heißt flach.
	\item[b)] Ein Vektorbündel $E$ mit flachem Zusammenhang heißt flaches Bündel.
\end{itemize}
\end{defs}

\begin{bsp}
2) \ Zusammenhang auf $\mfk \times \R^k$ gegeben durch eine $1$-Form mit Werten in $\operatorname{Mat}(k \times k,\R):$
\begin{align*}
\covd^{\omega}_X s &= X(\sigma) + \omega(X)\cdot\sigma \\
							&= \sum_{j=1}^{k} (X(\sigma^{j}) + \sum_{i=1}^{k} \sigma^{i}\omega_i^{\phantom{i}j}(X))s_j \\
							\phantom{.} \\
\curv^{\omega}(X,Y)s &= \covd^{\omega}_X(\covd^{\omega}_Ys) - \covd^{\omega}_Y(\covd^{\omega}_Xs) - \covd^{\omega}_{[X,Y]}s \\
&= \covd^{\omega}_X(Y(\sigma) + \omega(Y)\cdot\sigma) - \covd^{\omega}_Y(X(\sigma)+ \omega(X)\cdot\sigma) - [X,Y](\sigma) - \omega([X,Y]) \sigma \\
\overset{\text{Übung!}}{\phantom{=}} &= X(Y(\sigma)) + X(\omega(Y))\sigma - Y(X(\sigma)) - Y(\omega(X))\sigma + \omega(X)\omega(Y)\sigma - \omega(Y)\omega(X)\sigma - [X,Y]\sigma - \omega([X,Y])\sigma \\
&= \left( \underbrace{X(\omega(Y)) - Y(\omega(X)) - \omega([X,Y])}_{:= \dd\omega(X,Y)} - [\omega(X),\omega(Y)]\right) \sigma
\end{align*}
Hierbei wird $\dd$ äußeres Differential genannt.
\end{bsp}
\textbf{Lineare Zusammenhänge:} $ \quad E= T\mfk$ 
\begin{align*}
\nabla: \mathfrak{X}(\mfk) \times \mathfrak{X}(\mfk) \longrightarrow \mathfrak{X}(\mfk)
\end{align*}

\begin{align*}
\text{\textbf{Krümmung}}:  \qquad \qquad \qquad\curv: \mathfrak{X}(\mfk) \times \mathfrak{X}(\mfk) \times \mathfrak{X}(\mfk) &\longrightarrow \mathfrak{X}(\mfk) \\
\curv(X,Y)Z &= \nabla_X(\nabla_YZ) - \nabla_Y(\nabla_XZ) - \nabla_{[X,Y]}
\end{align*}


\begin{align*}
\text{\textbf{Torsion}:} \qquad\qquad\qquad\qquad\qquad \tor: \mathfrak{X}(\mfk) \times \mathfrak{X}(\mfk) &\longrightarrow \mathfrak{X}(\mfk) \\
(X,Y) &\longmapsto \tor(X,Y) := \nabla_XY -\nabla_YX - [X,Y]
\end{align*}

\begin{lem}
$\tor$ ist tensoriell in $X$ und $Y$.
\begin{bew}
$\tor(X,Y) = -\tor(Y,X)  \longrightarrow$ z.z. $\tor$ tensoriell in $X$. 
\begin{align*}
\tor(\varphi X,Y) &= \nabla_{\varphi X}Y -  \nabla_Y(\varphi X) - [\varphi X,Y] \\
						&= \varphi\tor(X,Y)
\end{align*}
\end{bew}
\end{lem}

\begin{defs}
Ein Zusammenhang $\nabla$ heißt symmetrisch, falls $\tor \equiv 0$.
\end{defs}

\missingfigure{Hier fehlt noch ein kleiner Teil :)}