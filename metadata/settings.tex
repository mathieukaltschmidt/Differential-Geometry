% !TeX root = ..//diffgeo_main.tex

%personel data
\title{Differentialgeometrie I}
\author{Dr. Anna Siffert}
\date{Sommersemester 2018}


%math and theorems
\usepackage{amsmath}
\usepackage{amssymb}
\usepackage{dsfont}
\usepackage[amsmath,thmmarks,hyperref]{ntheorem}
\usepackage{mathtools}
\usepackage{dsfont}
\usepackage[arrow, matrix, curve]{xy}
\usepackage{nicefrac}
\usepackage{wasysym}
\usepackage{upgreek} 

%language settings and microtype
\usepackage{fontspec} 
\setmainfont{Palatino}
\setsansfont{Optima}
\setmonofont[Scale=MatchLowercase]{Menlo}

\usepackage{polyglossia}
\setmainlanguage{german}
\setotherlanguages{english}
\usepackage{microtype}

%useful packages
\usepackage[dvipsnames]{xcolor}
\usepackage{graphicx}
\usepackage{float}
\usepackage{fancyhdr}
\usepackage{csquotes}
\usepackage{blindtext}
\usepackage{todonotes}
\usepackage{booktabs}
\usepackage{array}
\usepackage[labelfont=bf]{caption}
\usepackage{wrapfig}
\usepackage{physics}
\usepackage{enumitem}
\usepackage{xr} % cross referencing

%Page Layout and geometry
\usepackage{fancyhdr}
\fancyhfoffset{0pt}

\usepackage[a4paper,
			width = 150mm,
			top = 30mm,
		    bottom=30mm%,
		    %bindingoffset = 7mm
		    ]{geometry}

%Renewing marks for appearance in header
\renewcommand{\chaptermark}[1]{
    \markboth{\mbox{\@chapapp}\ \thechapter.\ \ #1}{}%
}
\renewcommand{\sectionmark}[1]{
    \markright{\thesection\ \ #1}{}
}

%Page Layout For Beginning Of Chapter
\fancypagestyle{plain}{%
	\fancyhf{}  %clear all header and footer fields
	\fancyfoot[C]{\thepage}
	\renewcommand{\headrulewidth}{0pt}
	\renewcommand{\footrulewidth}{0pt}
	}

%Page Layout For Other Pages
\pagestyle{fancy}
	\fancyhf{}
	\fancyhead[LE]{\footnotesize\nouppercase{\leftmark}}
	\fancyhead[RO]{\footnotesize\nouppercase{\rightmark}}
	\fancyfoot[C]{\thepage}
	\renewcommand{\headrulewidth}{0.2pt}
	\renewcommand{\footrulewidth}{0pt}

%Headings
\setkomafont{section}{\textsf\bfseries\Large}
\setkomafont{subsection}{\textsf\bfseries\large}
\setkomafont{subsubsection}{\textsf\bfseries\normalsize}



%color settings
\usepackage{xcolor}
\definecolor{myred}{RGB}{157,0,0}
\definecolor{myblue}{RGB}{1,1,141}
\definecolor{mygreen}{RGB}{0, 119, 85}
\definecolor{mygray}{RGB}{102, 102, 142}


%for empty pages at the beginning of the document
\def\blankpage{%
	\clearpage%
	\thispagestyle{empty}
	\addtocounter{page}{-1}
	\null%
	\clearpage}


\setkomafont{title}{\normalfont\bfseries\Large}
\setkomafont{author}{\normalfont\bfseries\Large}
\setkomafont{date}{\normalfont\bfseries\Large}

%Theorems
\theoremstyle{marginbreak}
\theoremheaderfont{\normalfont\bfseries}
\theorembodyfont{\slshape} 
\theoremseparator{}
\newtheorem{satz}{Satz}[chapter]

%Lemma
\theoremstyle{changebreak} 
\theoremindent0.5cm 
\theoremseparator{}
\newtheorem{lem}[satz]{Lemma}

%Helping lemma
\theoremstyle{changebreak} 
\theoremindent0.5cm 
\theoremseparator{}
\newtheorem{hlem}[satz]{Hilfslemma}

%Corolarys
\theoremindent0cm 
\theoremseparator{}
\newtheorem{kor}[satz]{Korollar}

%Examples
\theoremstyle{break} 
\theorembodyfont{\upshape} 
\theoremseparator{} 
\newtheorem{bsp}[satz]{Beispiel}

%Comments
\theoremstyle{plain} 
\theorembodyfont{\upshape} 
\theoremseparator{:} 
\newtheorem*{bem}[satz]{Bemerkung}

%Definitions
\theoremstyle{break}
\theorembodyfont{\slshape} 
\theoremseparator{} 
\newtheorem{defs}[satz]{Definition}

%Proofs
%\theoremheaderfont{\normalfont} %falls es nicht dick gedruckt sein soll
\theoremstyle{nonumberbreak}
\theoremseparator{} 
\theoremsymbol{\ensuremath{\square}}
\newtheorem{bew}[satz]{Beweis:}

%Bibliography
\usepackage[
	style=numeric-comp,
	backend=biber,
	isbn=false,
	date=year,
	url=false,
	doi=false,
	hyperref = auto
]{biblatex}
\addbibresource{bib/diffgeo-bib.bib}

%appendix
\usepackage[toc,page]{appendix}

%killing indent
\setlength{\parindent}{0pt}

%always use hyperref at the end of the preamble!
\usepackage[colorlinks=True]{hyperref}
\hypersetup{allcolors=myblue}
